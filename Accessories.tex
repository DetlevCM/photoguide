\chapter{Accessories}

This is one of the most difficult topics with photography, accessories. One can easily buy hundreds of items and then find they have not helped at all in improving one's photography. This is because accessories are not everything, however some are highly recommended.
\\
This chapter should give the reader some suggestions where to look and is held in a most general fashion to make it applicable to most users. In the end, only individual experience can determine which accessories you specifically require, there simply is no standard lists.
\\
\\
Having said that, please do not buy a pile of accessories beforehand just in case you need them, as you will waste your money. Buy your accessories step by step as you need them.

\section{Miscellaneous}

Little stuff that I highly recommended and cannot list under another category.
\begin{enumerate}[i]
	\item At least one spare battery.
	\item Enough flash memory cards to store a reasonable amount of photographs. Multiple cards are better than one in case one fails, however do not make them too small, especially considering the video capabilities available in modern SLRs. 4GB or 8GB cards are my suggestion.
	\item A microfibre cloth to clean lenses if they need a quick clean. Microfibre cloths can be had cheaply as household products or from an optician.
	\item If you are near or far sighted a correction lens for the \gls{viewfinder}.
\end{enumerate}

\section{Storage and Transport - Bags}

I personally like shoulder bags, some people prefer rucksacks and other use trolleys.
\\
Only you can decide what you want as a bag, but a couple of things you should look out for.

\begin{enumerate}[i]
	\item Does it have enough space if you add some accessories or one or two lenses? Buying a bag for only the equipment you have now will mean you need to buy a new one when you change your equipment.
	\item Is the material hard wearing? This may seem like a minor point, but you need a bag that will survive abuse. It is also protection for your camera, you do not want it falling apart.
	\item Is the bag well padded? If the padding is too thin it will not protect the contents well, but if it is too thick you will lose space.
	\item How heavy is the bag? Can you carry it for as long as you need to?
	\item If you buy a rucksack, can you attach a tripod to it.
\end{enumerate}

\section{Tripod and/or Monopod}

Tripods are one of the more general accessories that most photographers have a use for, often if they do not use a tripod they use a monopod.
\\
Tripods are essential for long \glspl{exposure}, practical for \gls{timelapse} videos or self portraits. If you are required to position the camera awkwardly, a tripod can also help to support it.
\\
\\
The following are a couple of points to consider when choosing a tripod.
\begin{enumerate}[i]
	\item Figure out the weight of your camera and consider any realistic upgrades.
	\\
	Keep in mind any super tele lenses or more professional SLRs you might use. An example woulld be upgrading from a 600D with the 55-200mm lens to a 7D with the 70-200mm f2.8. However do not overdo it. You do not need to consider say a medium format Hasselblad as by that time the 400-500\euro\ you would spend on a tripod to support that kind of weight will be small compared to the 10.000\euro\ or more price of such equipment. But spending another 150-200\euro\ because you followed the above mentioned realistic upgrade path is a waste of money. 
\\
Lastly, always keep a safety margin - so if your camera weights 3kg (5D MK II + Grip + 70-200mm 2.8), getting a head for at least 4kg is a good idea.
  \item Decide how much mobility you need from your tripod. Do you want a column that can be levelled horizontally, do you need multiple adjustment angles per leg? The more flexible a tripod, the more useful it becomes but also the more expensive it becomes. Also decide whether you want screw locks or snap locks on your legs. Snap locks can be fastened on good tripods should they become loose.
	\item Decide how much weight you are willing to carry. Carbon Fibre tripods are lighter than aluminium but a lot more expensive. The heavier the tripod, the cheaper. But heavy tripods are difficult to travel with.
Some people feel that carbon fibre tripods are more fragile than aluminium tripods. Personally I would go for aluminium because they are a lot cheaper.
	\item Decide on the height of the tripod - it needs to be comfortable to use, but if you are tall you need to compromise.
	\item Decide on what kind of head you want and consider that the head also has a weight limit just as the tripod.
	\subitem A video head for fluid movement?
	\subitem An "old fashioned" head for exact orientation (useful for panoramas).
	\subitem A ball head, easily quickly adjusted, but does not allow consistent changes,  such as one would use for a panorama.
\end{enumerate}

Numerous companies produce tripods, Manfrotto still produces in Europe as of writing this.
Other names are Giottos, Bogen. The choice is yours. Personally I use a Manfrotto 190 X Pro B with a 460MG head for a 5D MK II with a grip and 24-70 which weight about 2.4kg combined.

\section{Filters}

One of the most controversial topics possible. One side believe filters are a ``must have'' to ``protect'' lens elements, other had some bad experiences with ``always on'' filters.
\\
Personally, I would recommend you use a filter in only two conditions.
\begin{enumerate}
	\item You want the effect from it, such as a circular polarizer.
	\item You know you will be in an area wher hard particles may hit your lens, such as a desert, a ralley track or similar.
\end{enumerate}

Otherwise I would not recommend you buy a filter but rather inverst in a lens hood if anything.
\\
If you do decide to buy a filter, try to get high quality models. On circular polarizer the high quality filters let in a lot more light than the cheaper models, however both will achieve the same effect.
\\
I think if you are desperate, I suggest you go and research this topic by yourself.


\section{Flash and Diffusers}

Only you know whether you need a flash or not. The ``on-camera'' flash on lower SLR models is better than nothing if it can get you the shot, however it is nowhere near as good as a dedicated flash unit.
\\
If you feel you need to add light, a flash is often your only choice. If you buy a flash, consider the following points.

\begin{enumerate}
	\item Can the flash tilt and swivel the head?
	\item Do you have full manual control over the unit if you wish?
	\item Does it support ETTL-II (Canon) or a similar technology?
	\item Can it act as a slave or master.
	\item Is the output power sufficient? (It is better to have more power as you can always reduce the output but never increase it.)
\end{enumerate}

The slave/master point may seem like a minor one, however it is interesting in the long run. Buying a flash that can act at least as a slave allows you to later on buy a master and play with light creatively. If you buy a flash that cannot act as a slave you deprive yourself of that option unless you buy external triggers such as radio triggers.
\\
\\
In the end, flashes are a whole topic onto themself and I would highly recommend you research them in detail.
\\
If you do decide to buy one though, I suggest buying one with a swivel head, horizontally and vertically, which will allow you to bounce light off walls or ceilings as well as with a slave capacity should you decide to use a multi-flash setup in the future.
\\
\\
A diffusor is often a good advice too, these can be bought or self made from white plastic, I shall leave the choice unto you.