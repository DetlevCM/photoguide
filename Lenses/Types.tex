\section{Short Introduction to Different Types of Lenses}
\label{sec:Introduction-Types-Lenses}

\subsection{Prime versus Zoom}
\label{subsec:Prime-versus-Zoom}

Before I mention what types of lenses exist, I would like to introduce some more terminology. You are most likely familiar with a \gls{zoom} lens, which is a lens that can change its \gls{focallength}, or rather a lens that can cover multiple \glspl{focallength}.
\\[\baselineskip]
In contrast, a \gls{prime} lens is a lens that only covers one \gls{focallength}, which cannot be changed.
\\[\baselineskip]
If you ask, which is better, the answer is neither. A \gls{prime} lens will in general provide superior optical performance and offer wider apertures than a comparable \gls{focallength} \gls{zoom} lens. In contrast, a \gls{zoom} lens will offer a lot more versatility, covering a range of \glspl{focallength}.
\\
Some people use only \glspl{prime}, while others have a preference for \glspl{zoom}. You need to take your own decision on this topic. If you are not sure what to do, it may be an idea to buy a cheap 50mm \gls{prime} lens to go along with your kit lens and see which of the two lenses you prefer.

\subsection{Wide Angle}

Wide-angle lenses have a short \gls{focallength} and thus cover a wide angle of view.
\\
Classic uses for wide-angle lenses include landscape photography, architectural photography and interiors where space is at a premium. Portraits are generally not shot with a wide-angle lens as this results in a ``noisy'', distracting background and sometimes rather odd distortion. An exception to this is the photograph of a large group of people, especially if space is at a premium.

\subsection{Standard lenses}
Standard lenses cover a field of view roughly equal to the field of view covered by the human eye. They sit between tele lenses and wide-angle lenses and can be used in many different ways.
\\
Some people use standard lenses for portraits, others find the results unflattering. Some photographers use them for architectural photographs, others think they are too wide/narrow.
\\
Whether a standard zoom is right for you, can only be decided by yourself. It definitely does open the door to a lot of creative photography, as wide aperture standard lenses are comparatively cheap/reasonably priced.

\subsection{Tele}
Tele lenses are often used for portraits or photographs of details.
\\
With portraits, the narrow field of view allows for the visual isolation of the subject, while with for example architectural shots, tele lenses allow for shots of details that are otherwise out of reach.

\subsection{Super Tele}
In some cases, a ``normal'' tele lens is not enough, in this case super-tele lenses are used by photographers, which generally have a \gls{focallength} in excess of 200-300mm.
\\
Such lenses can be used for portraits, but are generally used to shoot sport events where the subject is far away as well as wildlife or aircraft. The primary objective behind super-tele lenses is reach, while still collecting enough light to allow for fast shutter speeds. With good optical quality, these lenses require large lens elements, which make them large and very expensive.
\\
While especially beginners might be tempted by cheap 70-300mm lenses from various manufacturers, their optical quality tends to be a so-so affair. I had a cheap 70-300mm f4.0-5.6 lens from Sigma, and while it was usable, and I got a couple of lovely shots from it, in retrospect I would not buy this lens again. At the same time, I got my lens in summer 2008, and since then lenses have become cheaper and optical qualities have been improved, on expensive as well as cheap lenses.

\subsection{Fisheye}
Fisheye lenses are a special type of wide-angle lens.
\\
While distortion is generally unwanted, fisheye lenses create specific distortion, similar to a convex mirror. This is paired with very large viewing angles of up to 180 degrees on some fisheye lenses, which means you will see your feet if you are not careful.
\\
Fisheye lenses have a primary use in artistic and creative photography. Only you can decide whether you have any use for them.

\subsection{Macro lenses}
Macro lenses are generally tele lenses with a very short \gls{minimumfocussingdistance}. Hence you are able to resolve a lot of detail in photographs of tiny objects.
\\
Macro lenses will also focus at infinity, so they do not have any disadvantages over ``normal'' tele lenses. In fact, you will find that macro lenses are typically just a part of the regular lens line-up without a ``non-macro equivalent''.

\subsection{Tilt-shift lenses}
Because conventional optics tend to lead to a ``leaning towers effect'' on buildings, architects required a lens that would show parallel lines/edges on buildings. The tilt-shift lens was born.
\\
By physically misaligning the lens elements, the photographer is able to ensure parallel lines on buildings. The disadvantage: tilt-shift lenses are not weather sealed and are manual focus only.
\\
Today they have also been discovered by artistic photographers who make creative use of the effects created by a tilt-shift lens.
