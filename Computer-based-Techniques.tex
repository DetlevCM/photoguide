\chapter{Computer based Techniques}
\label{chap:Computer-Based-Techniques}

Due to the nature of digital images and the constant advances of computing power, image manipulation has become a tool accessible to everybody.
\\
The two topics I would like to focus on in this chapter are \glspl{panorama} and \gls{HDR} images, both valuable techniques, which deserve their own chapter.

\section{HDR - High Dynamic Range}
\label{sec:HDR}

If you have read the earlier chapters, you should remember that the human eye can resolve about 18 \glspl{stop} of dynamic range, while a camera may at best resolve about 8 \glspl{stop} of dynamic range. In many scenes this is not a large issue, but some scenes loose all their magic as a result of losing highlight and shadow detail. The solution is to shoot an \gls{HDR} image, which consists of several images, covering the dynamic range of the scene, layered on top of each other to resolve the full dynamic range.
\\
The irony of this is, that while the image created during the post processing steps is a true high dynamic range image, the final result, though named ``high dynamic range'' is not as it is a compressed \gls{JPEG}.
\\[\baselineskip]
A most basic \gls{HDR} can be created from 2 images, but 3 or more are common. The most images I have ever used were over 40 images on a church window, to capture the interior as well as the beautiful stained glass.
\\
You may have guessed, that a tripod is vital when shooting \gls{HDR} images. Although an \gls{HDR} image can be created from images shot without a tripod, I would not recommend doing it at all, as your success rate will be extremely low.
\\[\baselineskip]
In most cases, you will want a series of images, from between 2-1 \glspl{stop} underexposed to 2-1 \glspl{stop} overexposed. However, again, only experience and ``on scene metering'' will give you an idea of the dynamic range and number of images you need.
\\[\baselineskip]
Creating an \gls{HDR} image is reasonably simple. The \gls{RAW} images are loaded into your ``\gls{HDR} creator'' of choice and you can await the result. Personally I use Photoshop CS4, but Oloneo is also great alternative that works with Photoshop. Many other utilities are available, from paid to free, such as Picturenaut, to open source, the choice is, eventually, yours.
\\[\baselineskip]
When using Photoshop, you need to convert your image from a 32bit colour depth to a 16bit colour depth to expose the tone mapping functionality. This is more easily accessible on other \gls{HDR} editors, and hence more intuitive. If you are using a photo editor other than Photoshop, such as Oloneo, edit your image until it ``looks right'' to you and export to your favourite format, if you aspire to do further editing, this should be a 16bit TIFF file or other lossless format with a high bit depth.. Also retain the settings and original file in case you want to get back to it later.

\section{HDR Examples}

Maybe it isn't yet quite clear to you why \gls{HDR} is a great tool. Let me help you visualize the advantages of \gls{HDR} with some examples.

\subsection{Doncaster Minster}

A church is a lovely building for photography, the history of centuries, elaborate ornaments, masterpieces of engineering and art. At the same time, the stained glass, is extremely difficult to photograph well if you want to be able to distinguish more than just the window or just the interior. A challenging, highly dynamic, scene that requires a technique such as \glspl{HDR}.
\\[\baselineskip]
The following 3 images are examples of the 45 images used to create the final image shown in figure \ref{fig:HDR1_1_16Bit}. I have chosen the shortest, ``middle'' and longest \gls{exposure} of the series for the example images. If you read the captions, you should spot that the exposure times are $\frac{1}{15}s$ for the image in figure \ref{fig:IMG_1091}, $1s$ for the image in figure \ref{fig:IMG_1113} and lastly $15s$ for the image in figure \ref{fig:IMG_1135}.
\\
Referring you back to the chapter on dynamic range, you should be able to spot that the difference between the shortest and longest \gls{exposure} is a factor of 225. ($\frac{15s}{\frac{1}{15}s} = 225$) This means that at the longest \gls{exposure}, the \gls{sensor} collected 225 times as much light as it did on the shortest \gls{exposure}, which translates to a difference in \gls{exposure} of about 7.8 \glspl{stop}. (7 \glspl{stop} are 128 times as much light, 8 \glspl{stop} are 256 times as much light).
\\
Of course, each image will in itself contain about 5-8 \glspl{stop} of dynamic range from the darkest to the brightest elements.
\\[\baselineskip]
Inspecting the images, you should see, and agree, that a single \gls{exposure} would not be able to capture the full dynamic range of the scene presented by the interior of Doncaster Minster.

\begin{figure}[htb]
\begin{minipage}{.32\textwidth}
	\centering
		\includegraphics[width=0.98\textwidth]{Images/HDR/Doncaster-Minster/IMG_1091.jpg}
	\caption{1/15 s}
	\label{fig:IMG_1091}
\end{minipage}
\begin{minipage}{.32\textwidth}
	\centering
		\includegraphics[width=0.98\textwidth]{Images/HDR/Doncaster-Minster/IMG_1113.jpg}
	\caption{1s}
	\label{fig:IMG_1113}
\end{minipage}
\begin{minipage}{.32\textwidth}
	\centering
		\includegraphics[width=0.98\textwidth]{Images/HDR/Doncaster-Minster/IMG_1135.jpg}
	\caption{15s}
	\label{fig:IMG_1135}
\end{minipage}
\end{figure}



\begin{figure}[htb]
	\centering
		\includegraphics[width=0.80\textwidth]{Images/HDR/Doncaster-Minster/HDR1_1_16Bit.jpg}
	\caption{Doncaster Minster - HDR}
	\label{fig:HDR1_1_16Bit}
\end{figure}

The final \gls{HDR} image, shown in figure \ref{fig:HDR1_1_16Bit} displays the difficult to capture dynamic range in the image beautifully. However, the name is deceiving, as the image you see on the screen is not an \gls{HDR} image, considering it is an 8bit \gls{JPEG}. What has happened is, that the 32bit \gls{HDR} image was compressed down to 16bit and then 8bit, which is a standard dynamic range image.

\subsection{Sunset in the Pennines}

Another scene which in its natural beauty is very dynamic is sunsets.
\\
It may not be intuitive to apply the \gls{HDR} technique to a sunset, but it has significant benefits for the resulting image. Because the sunset contains many highlight details, a photograph that captures the overall scene, does not capture the intricate details of the sunset.
\\[\baselineskip]
In this case, only three \glspl{exposure} were used:

\begin{figure}[htb]
\begin{minipage}{.32\textwidth}
	\centering
		\includegraphics[width=0.98\textwidth]{Images/HDR/Sunset-Pennines/IMG_4318.jpg}
	\caption{1/20 s, -1 stop}
	\label{fig:IMG_4318}
\end{minipage}
\begin{minipage}{.32\textwidth}
	\centering
		\includegraphics[width=0.98\textwidth]{Images/HDR/Sunset-Pennines/IMG_4317.jpg}
	\caption{1/10 s}
	\label{fig:IMG_4317}
\end{minipage}
\begin{minipage}{.32\textwidth}
	\centering
		\includegraphics[width=0.98\textwidth]{Images/HDR/Sunset-Pennines/IMG_4319.jpg}
	\caption{1/5 s, +1 stop}
	\label{fig:IMG_4319}
\end{minipage}
\end{figure}

It may not be obvious, but it is visible in figure \ref{fig:IMG_4318}, that the sunset contains some ``very bright'' spots, that require a short \gls{exposure}. At the same time, darker details on the clouds are better exposed in the images in figure \ref{fig:IMG_4317} and figure \ref{fig:IMG_4319}.
\\
An \gls{HDR} image allows the photographer to retain this detail in the clouds. Lastly, the additional information helps to improve the colours in the image. Just upping the saturation on a regular single photograph will not have the same effect. The resulting image can be seen in figure \ref{fig:3_2_8Bit}.

\begin{figure}
	\centering
		\includegraphics[width=0.80\textwidth]{Images/HDR/Sunset-Pennines/3_2_8Bit.jpg}
	\caption{Sunset in the Pennines - HDR}
	\label{fig:3_2_8Bit}
\end{figure}

\section{Panoramas}
\label{sec:Photomerge}

\Glspl{panorama} are merged images, mainly to resolve more detail than is possible with a conventional camera. It is also used to create resolution records, or to allow for a wider field of view than the lens permits, thought the latter should be used with caution.
\\[\baselineskip]
Again Photoshop can merge photographs, as can Microsoft ICE or other utilities, the choice is yours. I personally find ICE to be more accurate and definitely quicker than Photoshop.
\\[\baselineskip]
In terms of image overlap (which you need to merge images), a guideline is to have at least a $\frac{1}{3}$ overlap between two images. You can try with less, but it reduces the success rate and increases the chance of merging errors, experience will again be your best guide.
\\[\baselineskip]
If accuracy is your first concern, you should use a tripod to rotate your camera around fixed axis in space. Alternatively, you can try hand-holding, but be aware that the possibility of failure increases a lot with a handheld panorama. Overall I have had reasonable results when shooting handheld \glspl{panorama}, however a tripod is definitely a more failsafe method of creating a \gls{panorama}. If you are desperate for gigapixel images, a gigapan robot will be a worthwhile investment, but I must say that I have not used one myself.
\\[\baselineskip]
If you try to widen the field of view of a wide angle lens in confined space, for example indoors, this may lead to strange distortion effects, or if it is a symmetrical scene, a fisheye effect. So beware when creating a \gls{panorama}, it is not a perfect solution in every situation.
\\[\baselineskip]
Although the term \gls{panorama} traditionally refers to an image with a large aspect ratio, such as for example 16:9 or 21:9 over the standard 3:2 in photography, with the availability of merging tools to anybody, a \gls{panorama} is not tied to a specific format. In fact, because photographs may and are merged in any direction, the function in Photoshop is called ``\Gls{photomerge}'', rather than something along the lines of ``create \gls{panorama}''.

\subsection{Examples}

An example of distortion is the interior \gls{panorama} shot in De Nieuwe Kerk, Amsterdam, in figure \ref{fig:3-1}. However, as the distortion is symmetrical, it resembles a fisheye lens and is not too unpleasing. A distortion-free image would however be significantly better.

\begin{figure}[htb]
	\centering
		\includegraphics[width=0.80\textwidth]{Images/Panorama/3-1.jpg}
	\caption{De Nieuwe Kerk - distortion}
	\label{fig:3-1}
\end{figure}

An example for a perfect \gls{panorama} is the harbour in Mallaig, Scotland, shot in Winter 2008, shown in figure \ref{fig:pansmallaperture}. If I remember this image correctly, it consists of eight individual images in portrait mode.

\begin{figure}[htb]
	\centering
		\includegraphics[width=0.80\textwidth]{Images/Panorama/pansmallaperture.jpg}
	\caption{Mallaig Harbour at Christmas at night}
	\label{fig:pansmallaperture}
\end{figure}

The major difference between the image in figure \ref{fig:3-1} and figure \ref{fig:pansmallaperture}, is subject distance. In Mallaig the subject was significantly further away, and hence a longer focal length was used, resulting in a \gls{panorama} with less distortion.
