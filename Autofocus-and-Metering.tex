\chapter{Autofocus and Metering}


\section{Autofocus}
\label{sec:autofocus}

Whether you use \gls{AF} or \gls{MF} is really your choice, however nowadays, on digital \glspl{SLR}, autofocus is often more accurate. Exceptions again exist, and include product macro shots in a studio where manual focus tends to be preferable. Creative photography can also benefit from manual focus at times, especially when you have difficulty to find a suitable focussing target. In very dark environments, manual focus may also be beneficial, as the \glspl{AFsensor} require light to function properly.
\\
If you do use autofocus, learn how to control the focussing points (marked in the \gls{viewfinder}) or focussing point groups to your advantage. Do not let the camera decide what is important, but tell the camera where to focus automatically. While it is a great help to most photographers, it is a tool very easily abused, especially if misunderstood. The camera does not know what you want to focus on and thus how you want your image to look. Nevertheless, it is good at focussing on a spot, so use the autofocus to your advantage.
\\
\\
For this, you should know how the autofocus works.
\\
A camera has ``normal'' \glspl{AFsensor} or ``cross type'' \glspl{AFsensor}. (Double cross type exists, but is not commonly found on consumer grade equipment.) An \gls{AFsensor} measures the brightness change where it is pointed at. The greater the step in brightness, the less gradual the change, the sharper the image. An unfocussed image would result in a blurry, smooth change in brightness. This is why patterns are generally great for focussing and plain coloured flat areas are useless.
\\
\\
The following three images in figures \ref{fig:AF-Blur}, \ref{fig:AF-Blur-sharper} and \ref{fig:AF-Blur-focussed} give you a simplified example as to how an \gls{AFsensor} sees a ``target'' and what it aims to achieve.

\begin{figure}[htb]
\begin{minipage}{.32\textwidth}
	\centering
	\includegraphics[width=.98\textwidth]{Images/AF-Sensor/AF-Blur.jpg}
	\caption{unfocussed}
	\label{fig:AF-Blur}
\end{minipage}
\begin{minipage}{.32\textwidth}
	\centering
	\includegraphics[width=.98\textwidth]{Images/AF-Sensor/AF-Blur-sharper.jpg}
	\caption{badly focussed}
	\label{fig:AF-Blur-sharper}
\end{minipage}
\begin{minipage}{.32\textwidth}
	\centering
	\includegraphics[width=.98\textwidth]{Images/AF-Sensor/AF-Blur-focussed.jpg}
	\caption{focussed}
	\label{fig:AF-Blur-focussed}
\end{minipage}
\end{figure}

The difference between ``normal'' \glspl{AFsensor} and  a``cross type'' \gls{AFsensor} is, that a normal \gls{AFsensor} measures either horizontally or vertically while a cross type sensor measures horizontally and vertically. Pretty much every \gls{SLR} has at least one cross type \gls{AF} point at the centre. As a result, this is the most accurate individual focussing point on any \gls{SLR}. However, this does not mean in any way that the external points are bad. Again, it is a matter of choosing the right tools for the job and personal experience is most valuable in selecting the right \gls{AFpoint} or group.
\\
\\
Just for completion, should you have a double cross-type \gls{AFsensor}, it will measure vertically, horizontally and diagonally.


\section{Metering}

Metering is another important, or one could say vital, aspect of photography.
\\
You measure the light in a scene either with the ``needle'' in the viewfinder, which tends to typically be able to display $\pm$2 or $\pm$3 stops, or alternatively by using an external light meter.
\\
\\
The two key metering modes are spot metering or average metering. Spot metering measures the light at the active \gls{AFpoint}, average metering takes the whole scene into account.
Generally the camera will try to make whatever it meters (the spot or average brightness) an 18\% grey. This works on most scenes, but fails in snowy scenes (grey snow instead of white) or with black targets such as dark suits (which leads to overexposed images with grey suits). Again, experience is the best mentor in this case.
\\
On most landscape shots you will tend to want to be around $\pm$0 stops on the meter, but if you ever shoot in difficult and demanding conditions such as parties, concerts or night scenes, experience will tell you how to meter most effectively. Alternatively you can also just add or subtract from the displayed value using \gls{exposurecompensation} as appropriate and then again meter for $\pm$0.

\section{Histograms}

\subsection{Introduction}

One of the most valuable features on a camera is the histogram. While there is no perfect histogram, its look can give you an idea whether an image is over or underexposed.
\\
The height of the histogram, or rather the area under the peak represents the relative proportion of the pixels with the brightness denoted by the x-axis, going from black at the left ($x=0$) to white at the right. If the histogram touches either side, you have either lost shadow detail (touches the left side) or highlight detail (touches the right side).
\\
\\
You should also notice that the histogram on your (Canon) camera and in some editors is divided into 5 areas; these are the five \glspl{stop} of dynamic range that you will get in a typical \gls{JPEG} image. Mind you, a \gls{RAW} file, as mentioned earlier, can contain more than five \glspl{stop} of dynamic range and can be compressed into five \glspl{stop} in post processing.

\subsection{Examples}

So let us look at some examples. First, images as photographed, unprocessed to give you an idea as to how varied the histogram can be, then a look at some images and what histogram they present after they were edited.
\\
When the unedited RAW file is displayed, Canon's \gls{DPP} displays the light intensity in \glspl{stop}, when an edited image is shown, the colour channels are shown individually and without markings for stops of light.

\begin{figure}
	\centering
		\includegraphics[width=0.98\textwidth]{Images/Histogram/Day_Balanced.jpg}
	\caption{day, balanced}
	\label{fig:Day_Balanced}
\end{figure}

Take for example the histogram for the image in figure \ref{fig:Day_Balanced}, a fountain in the M\"{u}Ga in Germany. You should see a nice, not quite symmetrical, but ``right heavy'' overall dumbbell shape. The histogram touches neither the right hand side nor the left hand side at $-7.5$ or $4.0$. Hence neither shadow detail nor highlight detail has been lost.
\\
Overall, when shooting general scenes, a distribution such as this is what you should aim for, as it retains the largest amount of detail possible. However, one could argue that the image could be about 1 \gls{stop} brighter.

\begin{figure}
	\centering
		\includegraphics[width=0.98\textwidth]{Images/Histogram/Daylight_Light_Image.jpg}
	\caption{day, light image}
	\label{fig:Daylight_Light_Image}
\end{figure}

The next daylight image is presented in figure \ref{fig:Daylight_Light_Image}. You should again see that neither side of the histogram is touched, so no shadow or highlight detail has been lost. What you should further notice is, that the histogram touches the ``ceiling'' on the right hand side. This is of no concern, it just means that the count has gone ``through the roof'', hence a significant proportion of the image is very light or bright, but no detail has been lost. This can happen in bright scenes and as a result the overall histogram is a bit right heavy when compared to the histogram in figure \ref{fig:Day_Balanced}.

\begin{figure}
	\centering
		\includegraphics[width=0.98\textwidth]{Images/Histogram/Night_Balanced_Image.jpg}
	\caption{night, balanced image}
	\label{fig:Night_Balanced_Image}
\end{figure}

The next histogram is from a night scene and shown in figure \ref{fig:Night_Balanced_Image}. You should notice that is gives an overall balanced appearance. In fact, it is even slightly ``right heavy'' rather than ``left heavy'' as you would possibly have expected.
\\
This is not standard for night images, as figure \ref{fig:Night_Dark_Image} shows.

\begin{figure}
	\centering
		\includegraphics[width=0.98\textwidth]{Images/Histogram/Night_Dark_Image.jpg}
	\caption{night, dark image}
	\label{fig:Night_Dark_Image}
\end{figure}

Figure \ref{fig:Night_Dark_Image} shows the histogram of a more dynamic night scene. A well lit building, water that reflects light and the dark sky. The photographer will, unless it a \gls{HDR} approach is planned, have to make a compromise between capturing the details in darkness and the details well lit. If the darker scenes are correctly exposed, the light will be blown out white without detail, conversely, if the lamps are well exposed, the shadows will be absolute black.
\\
Overall however, the image is still reasonably well balanced, but more ``left heavy'', with more of the scene in darkness, but no obvious detail loss showing in the histogram. (Although I can tell you that the lamps and the wall behind them do not contain any detail.)
\\
\\
So what about images where shadow detail is lost? Take for example the following image from Leeds, shown in figure \ref{fig:Night}

\begin{figure}
	\centering
		\includegraphics[width=0.98\textwidth]{Images/Histogram/Night.jpg}
	\caption{night image}
	\label{fig:Night}
\end{figure}

The image is very ``left heavy'' with shadow detail lost. But does this make it a bad or wrong image? No, because the scene it depicts contains very dark areas. The underexposure was chosen to convey the mood of the scene and to maintain details in the light and reflections of light.

\subsection{Conclusion and Recommendation}

To conclude, there is no perfect histogram. No histogram is more right or wrong than another. However it is a useful tool to assess the exposure of an image quickly and to determine whether the resulting image has any problems due to the dynamic range of the scene captured.
\\
\\
Thus it is highly recommended that you check your histogram after capturing a scene. It is always accurate, no matter what lighting conditions and always more representative than the image visible on the LCD display.
\\
If you want to be extremely cautious, Canon cameras offer you the ability to display a histogram for every colour channel, as exposure can vary by a fair amount between different channels.
