\chapter*{Initial Words}


Many introductory books have been written on photography, one better than another and some worse. At the same time, is there not a lot of information in the manual of your SLR camera for free? At least in Canon's manual there is enough to understand the basics. For those who desire to know more, countless online resources, free and paid for, from videos to text offer more advice and information.
\\[\baselineskip]
So why another text?
\\
Mainly because I like to talk about photography and it would be nice to have a script to go with it. On the other hand, I like to ``play'' when I shoot, and playing is possibly one of the best ways of discovering photography. As I believe I have read somewhere ``first follow the rules, then break the rules'' or similar. 
But it begs the question, why follow the rules in the first place? If you aspire to become a wedding photographer, a photojournalist, yes, learn the rules, master the techniques, but if you see yourself as an artist? Learn the technique and play. If you are happy with the result, it is a result worth keeping, no matter what others think. Accept criticism, but make sure you are happy, make sure your vision was captured and ideally, know why the image was captured how it was captured.
\\[\baselineskip]
I must say, that I do not own a single book on photography. I have one year's worth of one photography magazine where some of the suggestions have been very valuable in getting into the basics of editing. At some point long in the past, I read the manual of my Canon SLR, but mainly, my knowledge is derived from the web and from playing. Of course talking with people was also valuable, both ``offline'' and online.
\\
The limit of what I know is how far I could be bothered to venture out. Hence I know as much a I want to know, and aim to share my knowledge in this document.
\\[\baselineskip]
I hope you will enjoy reading this script and find it useful.
\\[\baselineskip]
\textit{Detlev Conrad Mielczarek}


\section*{Author's Thanks}

Initially I wrote the text by myself - since however, I had the benefit of getting some input regarding corrections and improvements, in no particular order, the people who have contributed are listed below.

\begin{enumerate}[-]
	\item My thanks go to Kim Patrick O'Leary, a fellow photographer and also former editor, who proof read my script and pointed out quite a few corrections in the text. \\
His work is available at \url{http://www.patrickoleary.photoshelter.com}.
\end{enumerate}
