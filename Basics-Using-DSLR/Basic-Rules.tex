\section{Some basic Rules to Follow}

A \gls{DSLR} camera is a precision instrument, even if this is hard to believe considering how some professionals will treat their cameras. More amazingly though, they retain their precision in spite of the battering they receive. Still, it is recommended to follow a few guidelines set out by common sense and experience to extend life and hence joy you get out of your new camera.

\begin{enumerate}[i]
	\item Never touch the mirror in the camera. It is fragile. Maybe not as fragile as some people may make it seem, but you do not want to scratch it, let alone be forced to clean or replace it.
	
	\item Never store your \gls{SLR} left open, as this will allow dust to get in. Aim to changes lenses as seldom as possible, but as frequently as you need. Some people recommend changing lenses with the camera's mount facing down. If, for whatever reason you cannot store/transport your \gls{SLR} with the lens attached, use the supplied cover.
	
	\item Never use any force - if something does not come off or move easily, you are doing something wrong and possibly overlooked a release button somewhere.
	
	\item A camera only does what you tell it to do. Unless you get an error code of some sort, when it does not do what you want it to do, you have overlooked some setting(s).
\end{enumerate}