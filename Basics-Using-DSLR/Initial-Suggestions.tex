\section{Some initial Suggestions}

In general you can start to shoot with your new camera right away, however it is often advantageous to prepare it for its future use immediately, as well as preparing your computer for the editing it will face.

\begin{enumerate}[i]
	\item Install the manufacturer supplied software on your computer. Use this to enter your name in the owner field on the camera. Some modern cameras allow you to input this information on the camera itself, still install the software as it has further benefits. For Canon, the supplied \gls{DPP} is an ideal software for learning general basic editing skills.
	
	\item Change the camera setting so it will not take photographs without a memory card (\gls{CF} or \gls{SD}). This will be a custom function so consult your manual to determine which one it is, as this varies from camera to camera. It might seem not that important to you, but shooting without a memory card has happened to various photographers and is at least extremely frustrating. Changing this setting will prevent the potential  for any future grief caused by a lack of memory card.
	
	\item Just snap a few photographs on automatic using the \gls{viewfinder}. Even if your \gls{SLR} supports \gls{LiveView}, get into the habit of using the \gls{viewfinder}. Some exceptions exist were \gls{LiveView} is the better choice, but these are mainly studio environments in which product or macro-photography is practised. Or alternatively, when using a tripod, \gls{LiveView} also had some benefits. Your primary method of shooting photographs should be by looking through the \gls{viewfinder}.
	
	\item If you wear glasses, it is highly recommended the get a correcting lens for the eyepiece. It is not an absolute requirement, but it makes looking through the \gls{viewfinder} and obtaining an accurate perception of the scene easier.
	
	\item Have a look at the settings used on the photographs you snapped. An automatic mode just automatically chooses the settings for you, hence manual with the same settings would give you the same result. The manufacturer supplied software, which you should have installed on your computer, will display this information for you, including camera manufacturer specific information.
	
	\item Once you start to play and shoot, it is important that you use \gls{RAW} and not \gls{JPEG}. In some cases you may have a requirement for quick \gls{JPEG} images, for example if working as a \gls{photojournalist} at a sports event, but at the start you should aim to shoot only \gls{RAW}. Or, if you think that you have a requirement for a quick \gls{JPEG}.
\end{enumerate}

Before you continue to read, you should have a couple of images that may have turned out so-so or already looks quite well.
\\
Something to be said at this point is that photographing with a \gls{DSLR} is more work than with a compact camera. You are expected to post process your images, even the best you shoot. Even more so, if you shoot \gls{RAW} (which you generally should do).