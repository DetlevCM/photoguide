\section{Camera Modes}

Every \gls{DSLR} camera has a mode dial of some sort, an option to switch between automatic and manual modes. What I will be mentioning in this chapter will be building upon the naming scheme of Canon \glspl{DSLR}, however every major manufacturer incorporates the same features, just with different naming schemes.


\subsection{Automatic}
\label{subsec:automatic}

This mode is also referred to as ``green box mode'' on Canon \glspl{DSLR}. Generally frowned upon by photographers, it can give you decent to good results, but deprives you of understanding how an \gls{SLR} works or even how photography as a whole works. Additionally, the algorithmic implementation of functions such as \gls{AFpoint} slection might not agree with your expectations.
\\
If you have just bought your new camera, automatic is quite possibly a good mode to get some initial use out of your camera, but you should aim to move away from it as quickly as possible.
\\
Lastly, on older \glspl{SLR}, you might not be able to save images in \gls{RAW} when shooting in automatic mode, read section \ref{sec:shooting-RAW} to find out why shooting \gls{RAW} is important.


\subsection{P - Program}
\label{sec:P-Program}

The Program mode on Canon \glspl{SLR} is a first step up from automatic. For all intents and purposes, this is a fully automatic mode. However you are allowed to interfere with the camera's functions and settings. While \gls{exposure}, \gls{aperture} and \gls{ISO} are automatically, you are able to change these settings.
\\
\begin{enumerate}
	\item You can determine which focussing point the camera should use and do not have to leave it to chance/an algorithm that you do not know. Read section \ref{sec:autofocus} if you are interested in further details about the autofocus on a modern \gls{SLR}.
	
	\item You can dial in \gls{exposurecompensation}, measured in \glspl{stop}, hence you can decide whether your image should be brighter or darker, over- or underexposed.
	
	\item You can change the \gls{aperture}/\gls{shutterspeed} simultaneously, while the camera balances the \gls{exposure} according to your \gls{exposurecompensation}. However the camera will not retain your modification, when you refocus, it will again decide according to its own algorithm which \gls{exposure} and \gls{aperture} to use.
\end{enumerate}

Besides the three options mentioned above, you are further able to chose the metering mode, the colour temperature, the autofocus type and whether to shoot a single image or a series.


\subsection{Tv - Shutter Priority}

The Tv mode builds upon the P mode in section \ref{sec:P-Program}.
\\
However, you are given full control over the \gls{shutterspeed}. The camera will change the \gls{aperture} according to your \gls{exposurecompensation}. This is useful if you aim to capture for example action shots in difficult lighting where you require a certain \gls{shutterspeed}. Defining the \gls{shutterspeed} allows you to prevent motion blur, while staying focussed on shooting rather than changing your aperture to suit the lighting condition.
\\
More information on the \gls{shutterspeed}, also referred to as \gls{exposure} can be found in section \ref{sec:Exposure}


\subsection{Av - Aperture Priority}

The Av mode builds upon the P mode in section \ref{sec:P-Program}.
\\
However, you are given full control over the \gls{aperture} and the camera adapts the \gls{shutterspeed} according to your \gls{exposurecompensation}. This is useful for situations where you need to control the background blur, \gls{bokeh}, but face varying lighting conditions. It also has its use when you are in a situation where you do not have the time to manually change settings, but know you will require a specific minimum \gls{aperture}.
\\
More information on the \gls{aperture} can be found in section \ref{sec:Aperture}


\subsection{M - Manual}

The manual mode is what every photographer should aspire to be able to use competently.
\\
Manual stands, as the name says, for full manual control. You are given control over every setting, from \gls{aperture} over \gls{exposure} to \gls{ISO} as well as \gls{AF}, metering and colour temperature.
\\
However, having said that, I have seen people comment that the manual mode is pointless if all you do is zero the needle in the \gls{viewfinder}. This is true and you should chose the tools that suit your aims best. There is nothing wrong with using Av or Tv, provided you understand their workings and their limitations.
\\
To make full use of manual mode, you should read and understand chapter \ref{chap:Aperture-Exposure-ISO}.

\subsection{B - Bulb}

The bulb mode is for long \glspl{exposure} beyond 30s or photographs where you need the option of starting and stopping the \gls{exposure} manually . The \gls{shutterrelease} is pressed to start the \gls{exposure} and depressed to end it. However it would be very difficult to hold the camera steady if the button needs to be pressed for several seconds or even minutes. A tripod and remote control for your camera are thus highly recommended, which allow you to start the \gls{exposure} and end it without touching (and hence shaking the) camera.

\begin{figure}[htbp]
\centering
	\includegraphics[width=0.80\textwidth]{Images/Exposure-by-Stopwatch/_MG_5653.jpg}
	\caption{exposure of 106s}
	\label{fig:MG_5653}
\end{figure}

The image in figure \ref{fig:MG_5653} depicts one use for bulb mode, with an \gls{exposure} time of 106s, timed with a stopwatch.


