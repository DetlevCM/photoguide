\section{Explaining the Term DSLR}
\label{sec:Explaining-DSLR}

\Gls{SLR} refers to a camera type in which the viewfinder is used to look through the same lens which will be used to record the image. Hence, what you see (scene, framing) is what the \gls{sensor} will be exposed to. The term reflex specifically refers to the mirror that reflects the light into the \gls{viewfinder}, or rather onto the focussing screen if one wants to be accurate. The D was added when \gls{SLR} cameras went digital. Initially no match for film cameras in terms of quality, digital cameras have since beaten film in terms of \gls{ISO} and are close in terms of resolution (though some claim digital today has beaten film in this respect too).
\\
Estimates for the resolution of wet film vary wildly, with some people claiming around 10-15MP and some people claiming up 27MP on high quality low \gls{ISO}/ASA film.
\\
Of course, digital photographs were always more convenient. You can roughly evaluate your photograph ``on location'', you can transport photographs a lot more easily and lastly you no longer need a costly darkroom which requires experience. The darkroom may seem like an unobvious aspect, but a photography laboratory would only expose film for you, and charge you too. If you had any creative aspirations, there was no alternative to getting your own darkroom. With digital photography this is no longer an issue, as most people will have a computer. Even weak machines suffice, though they will take significantly longer for the same task when compared to a more powerful computer.
\\
Also, did I mention cost? Unless you are able to shoot very selectively, digital will be significantly cheaper than film which incurs a noticeable running cost. Digital storage is also reasonably cheap compared to a storing wet film, which requires a lot more space. \gls{SD}/\gls{CF} cards can be reused, while film may only be used once and then also comes with a ``use by'' date, a memory card does not ``go bad'' when stored. If I am to be picky, \gls{SD}/\gls{CF} cards have a limited lifetime in terms of write cycles, but the write cycles they will sustain are around several thousand per card (fully filled) and hence their cost becomes less important/insignificant in the grander scheme. I must say that none of my cards have so far gone bad from write cycles or in any other way for that matter.
\\
\\
So there are plenty of reasons to go digital nowadays, more so if you are just starting out in photography. Of course film retains its charm and has its followers, as well as certain advantages - but these do not outweigh the advantages of digital photography, especially if you are just starting.