\section{Exposure}
\label{sec:Exposure}

\Gls{exposure} indicates how long the sensor is exposed (hence the name) to light.
\\
A change in your \gls{exposure} by a factor of two equals a change in light by one \gls{stop}. Twice or half as much light will reach the sensor.
\\
A rule of thumb for photography without \gls{IS} or a tripod, is that $\frac{1}{focal\mbox{ }length\mbox{ }of\mbox{ }the\mbox{ }lens}$ is the longest \gls{exposure} you should use. So for example, on a 50mm lens, the maximum recommended \gls{exposure} when hand holding is $\frac{1}{50}$ seconds. This does not mean one cannot expose any longer, but it increases the risk of camera shake when not using \gls{IS}.
\\
As this rule dates back to the film era, and most \glspl{SLR} (for Canon all models except the 5D and 1Ds series) have a \gls{sensor} smaller than a slide of film, the maximum recommended \gls{exposure} when hand holding without \gls{IS} is actually shorter. For so-called \gls{APS-C} cameras from Canon, multiply by $1.6$ for Nikon by $1.5$. This means, a 50mm lens on an entry level Canon camera will have the field of view of an 80mm lens, and the maximum recommended \gls{exposure} is $\frac{1}{80}$ seconds.
(Just to be accurate, the 1D from Canon has an \gls{APS-H} \gls{sensor}, which needs a factor of $1.3$.)


\subsection{Creating Motion Blur}

\Gls{exposure} is important in two ways.
\\
On the one hand it describes how much light can reach the \gls{sensor}, on the other hand, it also controls whether an image contains motion blur, such as soft, flowing water in streams and waterfalls or a spinning propeller on an aircraft or helicopter or spinning wheels on a car.
\\
The longer the \gls{exposure}, the more motion blur, the shorter the exposure, the less motion blur. A propeller aircraft tends to show nice propeller blur at around $\frac{1}{250}s$ and anything faster than $\frac{1}{1000}s$ should freeze pretty much any motion.
\\[\baselineskip]
While not great images, the image in figure \ref{fig:IMG_5278} and figure \ref{fig:IMG_5353} illustrate the concept nicely. Notice how the propellers look different on the same King Air as it approaches the runway. On image \ref{fig:IMG_5278} the propellers are frozen, while on image \ref{fig:IMG_5353} the propellers show spin through motion blur resulting in a more pleasing image.

\begin{figure}[htb]
	\begin{minipage}{.5\textwidth}
		\includegraphics[width=0.98\textwidth]{Images/Shutter/IMG_5278.jpg}
		\caption{1/1250 s}
		\label{fig:IMG_5278}
	\end{minipage}
	\begin{minipage}{.5\textwidth}
		\includegraphics[width=0.98\textwidth]{Images/Shutter/IMG_5353.jpg}
		\caption{1/320 s}
		\label{fig:IMG_5353}
	\end{minipage}
\end{figure}


\subsection{Night Photography}

If you have not got a good tripod, I would highly recommend you obtain one. In many cases you may end up disappointed without a tripod, however night photography can be done with and without a tripod, either hand holding or using the support of a monopod.
\\[\baselineskip]
The question you need to ask yourself is, ``what is most important''. Is it getting the shot at all? Or is it getting a good/perfect result?
\\
If you aspire to become a photojournalist, getting the shot can make your career, in which case you would increase the \gls{ISO}, open the \gls{aperture} (See section \ref{sec:Aperture} for more on the \gls{aperture}) and do your best to use a \gls{shutterspeed} appropriate to your focal length. On the other hand, if your interest in photography is more artistic, you would aspire to obtain the best image quality possible, at a low \gls{ISO}, appropriate aperture and hence slow \gls{shutter}. However, this does not mean you must use a tripod at all times.
\\[\baselineskip]
For example, have a look at the following photographs, first figure \ref{fig:IMG_8581} and figure \ref{fig:IMG_8640}.
\\
The Humber Bridge might not look like a long \gls{exposure}, but it was a $30s$ \gls{exposure}. Even though I increased the \gls{ISO} a bit (see section {sec:ISO} for more details), because I used f13.0 I needed a long \gls{exposure}. As a result, the full image retains detail that would otherwise be lost in noise or by being out of focus. As the image file presented is a small sample, you will not be able to appreciate this fact that well. Further, the water has a silky smooth appearance because any waves or ripples are smoothed out by motion blur.
\\[\baselineskip]
The next image I would like to draw your attention to is presented in figure \ref{fig:IMG_8640}. A refinery at night, well lit. All the lighting allowed for a reduced \gls{exposure}, but again detail was key. The \gls{ISO} was lower when compared to the Humber Bridge, and the \gls{aperture} opened a bit more, still the \gls{exposure} time for the image was $5s$. You should be able to discern the smooth plumes of steam in the image. This is due to the long \gls{exposure}, which allowed the steam to travel, and smooth out details in it. A shorter \gls{exposure} would have just captured a point in time, including detail in the steam.
\\[\baselineskip]
Hence remember:
\begin{quote}
Exposing an image for a long amount of time leads to a smooth texture/blur on moving objects.
\end{quote}

\begin{figure}[htb]
\begin{minipage}{.5\textwidth}
	\centering
		\includegraphics[width=0.98\textwidth]{Images/Shutter/IMG_8581.jpg}
	\caption{Humber Bridge}
	\label{fig:IMG_8581}
\end{minipage}
\begin{minipage}{.5\textwidth}
	\centering
		\includegraphics[width=0.98\textwidth]{Images/Shutter/IMG_8640.jpg}
	\caption{Refinery}
	\label{fig:IMG_8640}
\end{minipage}
\end{figure}

After the long \gls{exposure}, the short \gls{exposure} night photography. Again two sample images that I would like to draw your attention to, this time one image from Leeds in figure \ref{fig:_mg_4686} and one image from Sheffield in figure \ref{fig:IMG_8859_1}.
\\[\baselineskip]
In Leeds City Centre, I did not have a tripod with me, but I really liked the reflections on the wet town square. I had walked past them without a camera, and there I was, with time, with a camera.
\\
I increased the \gls{ISO} as much as I was willing to do without getting to much noise, widened the \gls{aperture} and used an \gls{exposure} of $\frac{1}{40}s$ to capture this image. Could I have done better with a tripod? Of course. The question is, when will I be able to shoot such a scene again? I do not carry a camera with me every day, nor do I carry a tripod every time I carry a camera. Capturing an acceptable photograph in this moment was more important to me than getting the absolute best possible, which would be possible with a tripod.
\\[\baselineskip]
The next image I would like to draw your attention to is from the fountain at Sheffield Station, in figure \ref{fig:IMG_8859_1}. In this situation, setting up a tripod would be difficult, but not impossible. Additionally, I wanted to capture the water as frozen in time and not in motion. The image was shot at an elevated \gls{ISO}, with f4.0 at $\frac{1}{20}s$, handheld. A slow \gls{shutter} to allow enough light to reach the sensor, but also at the outer edges of what can be handheld steadily, especially in winter. The choice of \gls{shutterspeed} also allows for the water's flow to be frozen in time.

\begin{figure}[hthb]
\begin{minipage}{.5\textwidth}
	\centering
		\includegraphics[width=0.98\textwidth]{Images/Shutter/_mg_4686.jpg}
	\caption{Leeds City Centre}
	\label{fig:_mg_4686}
\end{minipage}
\begin{minipage}{.5\textwidth}
	\centering
		\includegraphics[width=0.98\textwidth]{Images/Shutter/IMG_8859_1.jpg}
	\caption{Sheffield, Fountain}
	\label{fig:IMG_8859_1}
\end{minipage}
\end{figure}

Whatever you chose in the end, it needs to gain you a result that you are happy with. I hope I have presented to you, that either approach, handholding or using a tripod can get you the desired photograph, but that using a tripod is beneficial if absolute image quality is your key concern.

\subsection{Conclusion - Exposure}

The best advice I can give you, is to go out and try it for yourself, as only experience will allow you to quickly judge the most appropriate \gls{shutterspeed}.
\\[\baselineskip]
One suggestion needs to be made though. If you plan on doing any night photography, get a good tripod, as even the best image stabilization generally does not help against camera shake at the\glspl{shutterspeed} frequently employed in night photography.
\\
The best \gls{IS} will gain you about 4 \glspl{stop} of light, while older lenses might just gain you about 2 \glspl{stop} with \gls{IS}.