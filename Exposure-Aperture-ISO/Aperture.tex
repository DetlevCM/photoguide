\section{Aperture}
\label{sec:Aperture}

\subsection{A Quick general Explanation}

\Gls{aperture} is important because it again has two functions, just like \gls{exposure}. It controls how much light can enter the camera, but also how much depth of field you get. Hence how much of the background/foreground is visible, and how much blurred.
\\[\baselineskip]
The following illustration in figure \ref{fig:Depth-of-Field} hopefully helps you to understand the concept of the depth of field.

\begin{figure}[htbp]
	\centering
		\includegraphics[width=0.80\textwidth]{Images/Aperture/Depth-of-Field/Depth-of-Field.png}
	\caption{Depth of Field - Illustration}
	\label{fig:Depth-of-Field}
\end{figure}

The yellow triangle in figure \ref{fig:Depth-of-Field} illustrates the angle of view through the lens, the camera being denoted by a black box. The red arrows refer to the distance to the subject, in this case a violet X.
\\
The blue area in figure \ref{fig:Depth-of-Field} is the depth of field. If we for example assume the ``X'' is 2 metres away, the depth of field can be $\pm5cm$, or $\pm10cm$. The depth of field depends on the \gls{aperture} and distance to the object, provided the \gls{aperture} and \gls{focallength} is constant. For a constant \gls{aperture}, the further the object, the greater the depth of field, the closer the object, the smaller the depth of field. As a result, you will find that macro photography often takes place at \glspl{aperture} such as f11 to ensure the object of interest is in focus.
\\
The effect of increasing depth of field with distance partially relates to infinity focus, which especially on wider lenses tends to begin as close a 3m or 10 feet. As a result using a wide angle to standard lens, an object 5m away would require nearly identical focussing compared to an object 6m away.
\\
For more detail on lenses, see chapter \ref{chap:Lenses}

\subsection{Mathematics of Stops with Aperture}

First the mathematics once again. A factor of the square root of two $\left(\sqrt{2}\right)$, which is approximately 1.41 equates to a difference of one \gls{stop}.
\\
Hence stopping a lens down from f2.0 to f2.8 would equate to a difference of one \gls{stop}. Stopping down from f2.8 to f4.0 would be another \gls{stop}. Stopping down from f2.8 to f3.2 would be a third of a \gls{stop}, while f2.8 to f3.5 would be two-thirds of a \gls{stop}. 

\subsection{Lenses and their Aperture}

A lens is described generally by its maximum aperture besides its focal length. For a ``kit lens'' this would typically be 18-55mm f3.5-5.6. This means, that at 18mm it has an \gls{aperture} of 3.5 and at 55mm an \gls{aperture} of 5.6, which translates to the lens letting in 1.3 \glspl{stop} less light at 55mm compared to 18mm. Such lenses are also called variable \gls{aperture} lenses.
\\[\baselineskip]
Lenses with a constant \gls{aperture} would be labelled for example 17-50mm f2.8, which denotes that the \gls{aperture} does not change with the focal length. A downside of these lenses is, that they are more difficult to build, requiring more glass, making them heavier and more expensive. Often however, they are also optically superior, especially when stopped down to match the \gls{aperture} of variable aperture lens. Most lenses perform at their worst when ``wide open'', hence at their maximum aperture, and benefit significantly from being stopped down one or two \glspl{stop}. The ``sweet spot'' of maximum sharpness is reached at f8 for many lenses, but this varies between different lenses.
\\[\baselineskip]
The largest \gls{aperture} commonly available on the market is f1.2 on professional Canon lenses, with Nikon possibly selling comparable products.
\\
Leica produces a very expensive f0.9 lens and Canon made a 50mm f1.0 in the past, where one copy I am aware of sold for \$5400 or more on EBay. Even an f1.2 lens will cost more than 1000\euro\ from Canon. The cheapest wide \gls{aperture} lens must be the Canon 50mm f1.8 lens at around 100\euro\. This should give you an estimate of lens pricing and as you can possibly gather, wide \gls{aperture} lenses are expensive.
\\
On the other end, most lenses will generally stop down to f22 and some tele lenses go as far down as f32. If you have a medium format camera, you will be able to find your lens stopping down even further.
\\[\baselineskip]
By now you possibly wonder why f32 is lower than f22 - well, technically the \gls{aperture} is a fraction and $\frac{1}{32}$ is smaller than $\frac{1}{22}$. This is confusing at the start, but just accepting the terminology is pretty much the only thing you can do. With time, one gets used to it. The bigger the ``f-number'', the smaller the \gls{aperture}.

\subsection{Background Blur - Bokeh}
\label{subsec:bokeh}

What I have mentioned, but have not elaborated on, was background blur. The technical term for it is ``\gls{bokeh}'' and comes from the Japanese. It has no direct translation and is used, as spelled, to describe a pleasant background blur. Some people view the \gls{bokeh} created by the 50mm f1.8 lens mentioned above as harsh, while the 85mm f1.2 bokeh is generally admired as one of the best. \Gls{bokeh}, or pleasing background blur, is a result of the shape of the aperture blades (pentagon shaped on the 50mm f1.8 or near perfectly circular or more expensive wide aperture lenses such as the 85mm f1.2) as well as the size of the \gls{aperture}.
\\
Generally, the background blur of lenses with circular aperture blades is viewed as more pleasing and hence described as the more pleasing \gls{bokeh}, but the quality of the glass used also has an effect.
\\[\baselineskip]
However, there is another element to background blur, namely the \gls{focallength}. A longer \gls{focallength} on a lens will result in a shorter depth of field at the same \gls{aperture}, compared to a lens with a shorter \gls{focallength}.
\\
Let me illustrate this with some examples. The distance rating is taken from the exif data in the photographs in figures \ref{fig:IMG_8095} and \ref{fig:IMG_8001} and estimated for the image in figure\ref{fig:IMG_8957}.

\begin{figure}[htbp]
\begin{minipage}{.35\textwidth}
	\centering
		\includegraphics[width=0.98\textwidth]{Images/Aperture/Example-Images/IMG_8095.jpg}
	\caption{f2.8, 70mm, 0.7m}
	\label{fig:IMG_8095}
\end{minipage}
\begin{minipage}{.28\textwidth}
\centering
		\includegraphics[width=0.98\textwidth]{Images/Aperture/Example-Images/IMG_8001.jpg}
	\caption{34mm, f2.8, 0.7m}
	\label{fig:IMG_8001}
\end{minipage}
\begin{minipage}{.35\textwidth}
	\centering
		\includegraphics[width=0.98\textwidth]{Images/Aperture/Example-Images/IMG_8957.jpg}
	\caption{300mm on 1.6 crop, f5.6, app. 4m}
	\label{fig:IMG_8957}
\end{minipage}
\end{figure}

While the \gls{aperture} for the photograph in figure \ref{fig:IMG_8957} was stopped down two \glspl{stop} over the \gls{aperture} settings for the photographs in figures \ref{fig:IMG_8095} and \ref{fig:IMG_8001}, the background blur is similarly pronounced, because a longer \gls{focallength} was used.
\\
Between the photographs in figures \ref{fig:IMG_8095} and \ref{fig:IMG_8001}, you should be able to determine less ``background blur'' in figure \ref{fig:IMG_8001} due to the reduced \gls{focallength}.

\subsection{Conclusion on Aperture}

The basic rule for beginners with respect to \gls{aperture} breaks down into the following two statements.

\begin{enumerate}[1]
	\item The wider the \gls{aperture} (the lower the f-number), the shallower the depth of field, hence the greater the background blur.
	\item Conversely, the smaller the \gls{aperture} (the bigger the f-number), the greater the depth of field and the background blur disappears/reduces.
\end{enumerate} 

If you desire an image with the foreground as sharp as distant objects, take a wide angle or standard zoom lens and set it to f22, manually focussing at 3 feet or 1m. 
