\section{ISO previously also ASA}
\label{sec:ISO}

\Gls{ISO}, or also ASA in the past, is used to denote how sensitive the film is to light. On a digital camera, it has an identical effect, but slightly different meaning. It does not so much describe sensitivity to light, but rather \gls{sensor} gain, which results in the same effect.
\\
A factor of two equates to a difference of one \gls{stop} with respect to \gls{ISO}.
\\
\\
As with the previous two settings, \gls{ISO} has a second side to it, mainly noise, or signal errors. The higher the \gls{ISO}, the greater the noise. It is often advisable to keep the \gls{ISO} setting low, but modern \glspl{SLR} should be very "clean" up to ISO800 across all manufacturers. Noise reduction can also help with noise, but increased \gls{ISO} also compresses the dynamic range. It is up to you to decide what \gls{ISO} settings you are happy to use on your camera, as every camera is different. Additionally, different situations as well as expectations to quality will be a part of your decision. For example, a noisy image is better than no image or a blurred image.


