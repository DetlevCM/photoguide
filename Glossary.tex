\newglossaryentry{aperture}{
name={aperture},
description={Diaphragm inside a lens which controls the depth of field and amount of light let in, described by f-stops. \\ More details can be found in section \ref{sec:Aperture}. \\ Page(s)}
}

\newglossaryentry{exposure}{
name={exposure},
description={The amount of time the sensor or film is exposed to light. \\ More details can be found in  section \ref{sec:Exposure}. \\ Page(s)}
}

\newglossaryentry{ISO}{
name={ISO},
description={A rating of film sensitivity or sensor gain. \\ More details can be found in section \ref{sec:ISO}. \\ Page(s)}
}

%% http://www.latex-community.org/index.php?option=com_content&view=article&id=263:glossaries-nomenclature-lists-of-symbols-and-acronyms
\newglossaryentry{HDR}{
name={HDR},
first={high dynamic range (HDR)},
description={A technique by which multiple images with varying exposures are used to capture a scene beyond the dynamic range of the sensor. \\ More details can be found in chapter \ref{chap:Computer-Based-Techniques}, section \ref{sec:HDR}. \\ Page(s)}
}

\newglossaryentry{photomerge}{
name={photomerge},
description={A technique by which multiple images are combined to form a single larger image. Photomerge is the specific term used by Photoshop for this function. Sometimes this is also referred to as a \gls{panorama}, because it is mainly used to create very wide images. \\ More details can be found in section \ref{sec:Photomerge}. \\ Page(s)}
}

\newglossaryentry{panorama}{
name={panorama},
description={A aspect ratio image, often the result of merging multiple images to form one large image. Also see  \gls{photomerge}. \\ More details can be found in section \ref{sec:Photomerge}. \\ Page(s)}
}

\newglossaryentry{colourmanagement}{
name={colour management},
description={Colour mamangement refers to managing the appearance of colours on screens and prints. \\ More details can be found in section \ref{sec:colour-management}. \\ Page(s)}
}

\newglossaryentry{bokeh}{
name={bokeh},
description={A term for the background blur created by lenses, coming from the Japanese. \\ More details can be found in section \ref{subsec:bokeh}. \\ Page(s)}
}

\newglossaryentry{colourspace}{
name={colour space}
,description={A predefined mapping of numerical values to specific colours. \\ More details can be found in section \ref{subsec:colourspace}. \\ Page(s)}
}

\newglossaryentry{DSLR}{name={DSLR},
first={digital single lens reflex (DSLR)},
description={A description for a camera class with interchangable lenses where the photographer looks through the lens. Contrary to an \gls{SLR}, a DSLR is digital. Sometimes it is spelt dSLR too. Generally, the term DSLR is used instead of DSLR camera. \\ More details can be found in section \ref{sec:Explaining-DSLR}. \\ Page(s)}
}

\newglossaryentry{SLR}{
name={SLR},
first={single lens reflex (SLR)},
description={A description for a camera class with interchangable lenses where the photographer looks through the lens. Generally, the term SLR is used instead of SLR camera. \\ More details can be found in the \gls{DSLR} section \ref{sec:Explaining-DSLR}. \\ Page(s)}
}

\newglossaryentry{exposurecompensation}{
name={exposure compensation},
description={A feature in \gls{SLR} cameras, which allows the user to purposely over or underexpose a photograph by a predefined amount, described in \gls{stop}.
%\\ More details can be found in section \ref{}.
 \\ Page(s)}
}

\newglossaryentry{stop}{
name={stop},
description={A relational measure of light. One stop difference equals half as much or twice as much light. \\ More details can be found in section \ref{subsec:Camera-Sensor-Human-Eye}. \\ Page(s)}
}

\newglossaryentry{RAW}{
name={RAW},
description={A generic term for raw sensor data. Being uncompressed (or losslessly compressed) and unprocessed, this retains the largest amount of information possible for further use. \\ More details can be found in section \ref{sec:shooting-RAW}. \\ Page(s)}
}

\newglossaryentry{JPEG}{
name={JPEG},
description={A lossy image compression algorithm which is widely adapted and supported. JPEG stands for ``Joint Photographic Experts Group''. \\ More details can be found online \href{http://www.jpeg.org/}{www.jpeg.org/}. \\ Page(s)}
}

\newglossaryentry{DPP}{
name={DPP},
first={Digital Photo Professional (DPP)},
description={Canon's basic but high quality \gls{RAW} interpreted/converter, supplied for free with RAW-capable Canon cameras. \\ Page(s)}
}

\newglossaryentry{shutter}{
name={shutter},
description={A curtain that controlls how long the camera's sensor is exposed to light, by travelling in front of the sensor. \Gls{exposure} is controlled by the shutter, hence also ther term \gls{shutterspeed} to describe \gls{exposure} time. \\ More details can be found in section \ref{sec:Exposure}. \\ Page(s)}
}

\newglossaryentry{shutterspeed}{
name={shutter speed},
description={Another way of referring to \gls{exposure}. \\ Page(s)}
}

\newglossaryentry{CF}
{name={CF},
first={Compact Flash (CF)},
description={An old memory card format that has received constant backward compatible specification upgrades. CF stands for Compact Flash The fastest memory cards on the market are CF cards. \\ Page(s)}
}

\newglossaryentry{SD}{name={SD},
description={A memory card format that became popular in \gls{PointShoot} cameras. Nowadays entry level \glspl{DSLR} have started to use it too in favour of \gls{CF} cards. \\ Page(s)}
}

\newglossaryentry{PointShoot}{
name={Point \& Shoot},
description={Description for a compact digital camera class which is fully automated. \\ Page(s)}
}

\newglossaryentry{AF}{
name={AF},
first={auto focus (AF)},
description={Automatic focus refers to a setting/mode where the camera focusses the image, as opposed to \gls{MF}. This is done with the help of an \gls{AFsensor} and a motor in the lens or body. \\ Page(s)}
}

\newglossaryentry{MF}{
name={MF},
first={manual focus (MF)},
description={Manual focus refers to a setting/mode where the user has to focus the image in the viewfinder/on \gls{LiveView} by hand. \\ Page(s)}
}

\newglossaryentry{LiveView}{
name={LiveView},
description={LiveView refers to a setting in which the image seen by a sensor is displayed on the back \gls{LCD} screen, rather than in the viewfinder. When in LiveView, the \gls{viewfinder} is blocked by the mirror an cannot be used. \\ Page(s)}
}

\newglossaryentry{LCD}{
name={LCD},
first={liquid crystal display (LCD)},
description={A display technology for flat screen panels. \\ Page(s)}
}

\newglossaryentry{viewfinder}{
name={viewfinder},
description={A window that offers the same field of view as the camera lens. On an \gls{SLR} the \gls{viewfinder} looks through the lens, on a \gls{rangefinder} or older compact camera, the \gls{viewfinder} functions indepednently of the main lens. . \\ Page(s)}
}

\newglossaryentry{rangefinder}{
name={rangefinder},
description={An interchangeable lens camera, where the \gls{viewfinder} is used to estimate the distance to the subject. This information is then used to focus the lens. \\ The Leica M8 and M9 are iconic rangefinders and besides the Fujifilm X10 the only models availabel on the market at the point of writing in December 2011. \\ Page(s)}
}

\newglossaryentry{sensor}{
name={sensor},
description={A device to measure something. In the context of cameras, this refers to measuring brightness in the RGB colour channels. Either how sudden the change is in the \gls{AF} sensor, or the intensity \\ Page(s)}
}

\newglossaryentry{AFsensor}{
name={AF sensor},
description={Same as \gls{sensor}, but specifcially designed to to measure a change in contrast. The bigger the change, the sharper the image. Older AF sensors measured in the red channel only, while newer cameras make use of multiple colour channels. \\ Page(s)}
}

\newglossaryentry{AFpoint}{
name={AF point},
description={A specific point, marked in the \gls{viewfinder} of an \gls{SLR}, at which the light is directed to an \gls{AFsensor} for focussing. \\ More details can be found in section \ref{sec:autofocus}. \\ Page(s)}
}

\newglossaryentry{photojournalist}{
name={photojournalist},
description={A photographer who shoots primarly to document events, hence shoots photographs for journalistic use or documentary purposes. As a result, it is required that these images are edited very little, or not edited at all. \\ Page(s)}
}

\newglossaryentry{prime}{
name={prime},
description={A prime, also prime lens, is a lens with a fixed focal length. A key benefit of primes is their wider \gls{aperture} and often superior optical quality over \gls{zoom} lenses. \\ More details can be found in section \ref{subsec:Prime-versus-Zoom}. \\ Page(s)}
}

\newglossaryentry{zoom}{
name={zoom},
description={A zoom, also zoom lens, is a variable focal lenght lens. \\ More details can be found in section \ref{subsec:Prime-versus-Zoom}. \\ Page(s)}
}

\newglossaryentry{focallength}{
name={focal lenght},
description={If parallel light were to enter a lens element, the focal length is the distance to the \gls{focalpoint}. In simple lens elements, this is measured from the centre of the lens element, however in camera lenses this is more difficult. \\ Page(s)}
}

\newglossaryentry{focalpoint}{
name={focal point},
description={The point at which parallel light entering a lens element is bundled to meet in a single point. \\ Page(s)}
}

\newglossaryentry{minimumfocussingdistance}{
name={minimum focussing distance},
description={The minimum distance from the frontal lens element to the subject at which the lens will still focus. This is sometimes abbreviated to mfd. \\ Page(s)}
}

\newglossaryentry{fullframesensor}{
name={full frame sensor},
description={A full frame sensor is a \gls{sensor} which has the same physical dimentions as a slide of film, namely $36mm \times 24mm$. Because of its size, a full frame sensor is significantly more expensive than a \gls{cropsensor}. \\ Page(s)}
}

\newglossaryentry{fullframe}{
name={full frame},
description={Full frame is the short form for \gls{fullframesensor}. \\ Page(s)}}

\newglossaryentry{cropsensor}{
name={crop sensor},
description={A crop sensor is a sensor that is smaller than \gls{fullframe}, mainly to save costs. Crop sensors are defined by the \gls{cropfactor}, which describes their size in relation to \gls{fullframe}. \\ Page(s)}
}

\newglossaryentry{crop}{
name={crop},
description={Crop is the short form for \gls{cropsensor} or refers to the \gls{cropfactor}\\ Page(s)}
}

\newglossaryentry{cropfactor}{
name={crop factor},
description={A factor which defines the size of a \gls{cropsensor} in relationship to a \gls{fullframesensor}. Multiplying the lenght of the diagonal of the \gls{cropsensor} with the cropfactor will give the length of the diagonal of a \gls{fullframesensor}. \\ Addtionally, the crop factor describes the change in the field of view resulting from the smaller \gls{sensor}. \\ More details can be found in section \ref{sec:General-Introdcution-Lenses}. \\ Page(s)}
}

\newglossaryentry{APS-C}{
name={APS-C},
description={The name for Canon's \gls{sensor} size with a \gls{cropfactor} of 1.6. \\ Page(s)}
}

\newglossaryentry{APS-H}{
name={APS-H},
description={The name for Canon's \gls{sensor} size with a \gls{cropfactor} of 1.3. \\ Page(s)}
}

\newglossaryentry{FF}{
name={FF},
description={Mainly used on the web, FF is the abbreviation for \gls{fullframe}. \\ Page(s)}
}

\newglossaryentry{IS}{
name={IS},
first={image stabilization (IS)},
description={IS stands for Canon's ``image stabilization'' technology built into lenses. Also see \gls{VR} for Tamron. \\ Page(s)}
}

\newglossaryentry{VR}{
name={VR},
first={vibration control (VR)},
description={VR is Tamron's name for ``vibration control'', which Canon calls \gls{IS}. \\ Page(s)}
}

\newglossaryentry{EF}{
name={EF},
description={Name of a Canon mount that fits all cameras, hence \gls{APS-C}, \gls{APS-H} and \gls{FF} \glspl{sensor}.  \\ Page(s)}
}

\newglossaryentry{EF-S}{
name={EF-S},
description={Name of a Canon mount that will only fit \gls{crop} cameras with an \gls{APS-C} \gls{sensor}. Because these lenses need to throw a smaller \gls{imagecircle}, they tend to contain less or smaller glass and hence are lighter as well as cheaper.\\ Page(s)}
}

\newglossaryentry{imagecircle}{
name={image circle},
description={The image circle, describes the diameter of the circular image thrown by the lens onto the \gls{sensor}. \\ Page(s)}
}

\newglossaryentry{shutterrelease}{
name={shutter release},
description={The button you depress to expose a single image or a series of images. \\ Page(s)}
}

\newglossaryentry{timelapse}{
name={time lapse},
description={Time lapse refers to shooting a set of photographs over a significant amount of time and then combining them to a video that depicts the change in for example light at a sunset. Typically timelapse videos cover at least minutes but often hours. \\ Page(s)}}

%\newglossaryentry{}{name={},description={ \\ More details can be found in section \ref{}. \\ Page(s)}}

%\newglossaryentry{}{name={},description={ \\ More details can be found in section \ref{}. \\ Page(s)}}

%\newglossaryentry{}{name={},description={ \\ More details can be found in section \ref{}. \\ Page(s)}}

%\newglossaryentry{}{name={},description={ \\ More details can be found in section \ref{}. \\ Page(s)}}

%\newglossaryentry{}{name={},description={ \\ More details can be found in section \ref{}. \\ Page(s)}}

%\newglossaryentry{}{name={},description={ \\ More details can be found in section \ref{}. \\ Page(s)}}

%\newglossaryentry{}{name={},description={ \\ More details can be found in section \ref{}. \\ Page(s)}}

%\newglossaryentry{}{name={},description={ \\ More details can be found in section \ref{}. \\ Page(s)}}

%\newglossaryentry{}{name={},description={ \\ More details can be found in section \ref{}. \\ Page(s)}}

%\newglossaryentry{}{name={},description={ \\ More details can be found in section \ref{}. \\ Page(s)}}

%\newglossaryentry{}{name={},description={ \\ More details can be found in section \ref{}. \\ Page(s)}}

%\newglossaryentry{}{name={},description={ \\ More details can be found in section \ref{}. \\ Page(s)}}

%\newglossaryentry{}{name={},description={ \\ More details can be found in section \ref{}. \\ Page(s)}}

%\newglossaryentry{}{name={},description={ \\ More details can be found in section \ref{}. \\ Page(s)}}

%\newglossaryentry{}{name={},description={ \\ More details can be found in section \ref{}. \\ Page(s)}}

%\newglossaryentry{}{name={},description={ \\ More details can be found in section \ref{}. \\ Page(s)}}

%\newglossaryentry{}{name={},description={ \\ More details can be found in section \ref{}. \\ Page(s)}}

%\newglossaryentry{}{name={},description={ \\ More details can be found in section \ref{}. \\ Page(s)}}

%\newglossaryentry{}{name={},description={ \\ More details can be found in section \ref{}. \\ Page(s)}}

%\newglossaryentry{}{name={},description={ \\ More details can be found in section \ref{}. \\ Page(s)}}

%\newglossaryentry{}{name={},description={ \\ More details can be found in section \ref{}. \\ Page(s)}}

%\newglossaryentry{}{name={},description={ \\ More details can be found in section \ref{}. \\ Page(s)}}

%\newglossaryentry{}{name={},description={ \\ More details can be found in section \ref{}. \\ Page(s)}}

%\newglossaryentry{}{name={},description={ \\ More details can be found in section \ref{}. \\ Page(s)}}

%\newglossaryentry{}{name={},description={ \\ More details can be found in section \ref{}. \\ Page(s)}}

%\newglossaryentry{}{name={},description={ \\ More details can be found in section \ref{}. \\ Page(s)}}

%\newglossaryentry{}{name={},description={ \\ More details can be found in section \ref{}. \\ Page(s)}}

%\newglossaryentry{}{name={},description={ \\ More details can be found in section \ref{}. \\ Page(s)}}

%\newglossaryentry{}{name={},description={ \\ More details can be found in section \ref{}. \\ Page(s)}}

%\newglossaryentry{}{name={},description={ \\ More details can be found in section \ref{}. \\ Page(s)}}

%\newglossaryentry{}{name={},description={ \\ More details can be found in section \ref{}. \\ Page(s)}}

%\newglossaryentry{}{name={},description={ \\ More details can be found in section \ref{}. \\ Page(s)}}

%\newglossaryentry{}{name={},description={ \\ More details can be found in section \ref{}. \\ Page(s)}}

%\newglossaryentry{}{name={},description={ \\ More details can be found in section \ref{}. \\ Page(s)}}

%\newglossaryentry{}{name={},description={ \\ More details can be found in section \ref{}. \\ Page(s)}}

%\newglossaryentry{}{name={},description={ \\ More details can be found in section \ref{}. \\ Page(s)}}

%\newglossaryentry{}{name={},description={ \\ More details can be found in section \ref{}. \\ Page(s)}}

%\newglossaryentry{}{name={},description={ \\ More details can be found in section \ref{}. \\ Page(s)}}

%\newglossaryentry{}{name={},description={ \\ More details can be found in section \ref{}. \\ Page(s)}}

%\newglossaryentry{}{name={},description={ \\ More details can be found in section \ref{}. \\ Page(s)}}

%\newglossaryentry{}{name={},description={ \\ More details can be found in section \ref{}. \\ Page(s)}}

%\newglossaryentry{}{name={},description={ \\ More details can be found in section \ref{}. \\ Page(s)}}

%\newglossaryentry{}{name={},description={ \\ More details can be found in section \ref{}. \\ Page(s)}}

%\newglossaryentry{}{name={},description={ \\ More details can be found in section \ref{}. \\ Page(s)}}

%\newglossaryentry{}{name={},description={ \\ More details can be found in section \ref{}. \\ Page(s)}}

%\newglossaryentry{}{name={},description={}}
