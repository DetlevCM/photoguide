\section{Key Steps for Post Processing}

The most basic post processing steps that you should consider are

\begin{enumerate}
	\item White balance - is the colour temperature right or not?
	
	\item Exposure - is the image exposed correctly?
	
	\item Highlights and shadows - do they show how you want them to show; you have not lost any detail that you want to keep?
	
	\item Saturation - does the image look dull?
\end{enumerate}

While you might hope there is an ``always right and applicable'' rule to editing photographs, I must say that there is not. Any steps you undertake when editing a photograph represent your evaluation of the scene and your interpretation. While there are settings you should generally avoid, there is no perfect solution when editing images.
\\[\baselineskip]
If you do not have a clear idea where you want to go when editing the image, I would highly recommend you aim for a natural look in your photographs.
\\[\baselineskip]
In the following images, I would like to introduce you to some visualisations of key settings. Figure \ref{fig:New} shows an unedited conversion of the \gls{RAW} file depicting the scene. Figure \ref{fig:IMG_5335} shows my interpretation of the scene, which I can tell you, is a sunrise in Sheffield.
\\
Specifically, I increased the colour temperature (made the image warmer), increased the saturation and pushed (hence brightened) the shadows a bit.

\begin{figure}[htbp]
\begin{minipage}{.5\textwidth}
	\centering
		\includegraphics[width=0.98\textwidth]{Images/Editing/Key-Steps/New.jpg}
	\caption{initial file}
	\label{fig:New}
\end{minipage}
\begin{minipage}{.5\textwidth}
	\centering
		\includegraphics[width=0.98\textwidth]{Images/Editing/Key-Steps/IMG_5335.jpg}
	\caption{my interpretation}
	\label{fig:IMG_5335}
\end{minipage}
\end{figure}


\subsection{Colour Temperature}

Colour temperature describes how warm or cold light, or an image is. A low colour temperature corresponds to a cold or blueish image while a high colour temperature corresponds to a warm or yellow/orange image.
\\
Colour temperature is measured in Kelvin, K.
\\[\baselineskip]
The image in figure \ref{fig:New(1)} shows a temperature of 2000K, while the image in figure \ref{fig:New(Kopie)} shows a colour temperature of 50000K. In most typical photographs you will be looking at a range of 4000-7000K on the extremes, however, just like many other settings, this is such an individual setting that only you can chose the appropriate colour temperature for your image.

\begin{figure}[htbp]
\begin{minipage}{.5\textwidth}
	\centering
		\includegraphics[width=0.98\textwidth]{Images/Editing/Key-Steps/New(1).jpg}
	\caption{cold colour temperature}
	\label{fig:New(1)}
\end{minipage}
\begin{minipage}{.5\textwidth}
	\centering
		\includegraphics[width=0.98\textwidth]{Images/Editing/Key-Steps/New(Kopie).jpg}
	\caption{colour temperature too warm}
	\label{fig:New(Kopie)}
\end{minipage}
\end{figure}


\subsection{Digital Exposure}

Digital \gls{exposure} works very much like ``real \gls{exposure}'', adding or removing \glspl{stop} or fractions of \glspl{stop} of light. However, contrary to when you are exposing the actual \gls{sensor}, increasing the \gls{exposure} in post processing adds noise, while reducing the \gls{exposure} can not always recover lost highlights.
\\
How far you can go either way depends on many factors, amongst them the camera's age and manufacturer. Experience is again your best guide in this respect. However, as a general rule, you should aim to expose as perfectly as possible in the camera, as this will always yield the best results in the final image. Correctly exposing in the first place is nearly always better than adding \gls{exposure} later on in post processing
\\[\baselineskip]
The image in figure \ref{fig:New(2)} depicts one extreme, with four \glspl{stop} of light added in post processing. As a result the station is clearly visible, but the sky is just plain white. Conversely, the image in figure \ref{fig:New(3)} has four \glspl{stop} of light removed from it, resulting in a station which is absolute black but cloud detail visible.
\\[\baselineskip]
When you edit your photographs, you will want to balance the \gls{exposure}, shadow and highlights slider to try to find an ideal solution. The shadow slider will brighten up the underexposed parts of the image only, while the highlights slider will only darken then brighter parts of the image.
\\
Please note that the naming convention depends on the manufacturer/supplier of the software. Canon's DPP calls them ``shadow'' and ``highlight'', while Adobe's CameraRAW calls them ``Fill Light'' and ``Repair''.

\begin{figure}[htbp]
\begin{minipage}{.5\textwidth}
	\centering
		\includegraphics[width=0.98\textwidth]{Images/Editing/Key-Steps/New(2).jpg}
	\caption{severely overexposed}
	\label{fig:New(2)}
\end{minipage}
\begin{minipage}{.5\textwidth}
	\centering
		\includegraphics[width=0.98\textwidth]{Images/Editing/Key-Steps/New(3).jpg}
	\caption{severely underexposed}
	\label{fig:New(3)}
\end{minipage}
\end{figure}


\subsection{Saturation}

Saturation is, in a rough description, the intensity of colour. You may know that a completely desaturated image will be a black and white image, shown in figure \ref{fig:New(5)}. If you oversaturate an image, the colours will look very brilliant as seen in figure \ref{fig:New(4)}, but also loose highlight detail and the image will look unnatural.
\\[\baselineskip]
An image can be desaturated or (over)saturated for artistic purposes. You need to decide what looks good in your eyes and what you think fits the image.

\begin{figure}[htbp]
\begin{minipage}{.5\textwidth}
	\centering
		\includegraphics[width=0.98\textwidth]{Images/Editing/Key-Steps/New(4).jpg}
	\caption{overly saturated}
	\label{fig:New(4)}
\end{minipage}
\begin{minipage}{.5\textwidth}
	\centering
		\includegraphics[width=0.98\textwidth]{Images/Editing/Key-Steps/New(5).jpg}
	\caption{desaturated}
	\label{fig:New(5)}
\end{minipage}
\end{figure}


\section{Further Steps for Post Processing}

Further steps are contrast (via tonal curves) and sharpening. Do not over-sharpen the image, even if you are used to this look from compact cameras. Together with sharpening, you will also want to apply some noise reduction, but only in a manner such that noise is reduced while detail is not lost excessively. Again, experience is valuable, and experimenting with different settings is highly recommended.