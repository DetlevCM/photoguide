\section{Introduction to Editing}

Editing photographs becomes an essential part of photography, the only exception being for example sports photography where photographers compete to have their image available first and hence do not have time to edit photographs.
\\
Even in photojournalism, most situations will require a little bit of post processing (editing), however only in a manner such that it does not distort reality. Hence cropping, sharpening and minor colour corrections would be acceptable, but significant changes such as digital filters or significant colour changes would not be.
\\
If however, your interest lies with photography as an art, you are free to do whatever you want.
\\[\baselineskip]
Before you start, please ensure you do not edit your only photograph destructively. Programs such as CameraRAW can edit images non-destructively, by writing changes into a separate file. Canon's Digital Photo Professional is also non-destructive on \gls{RAW} files. However, Photoshop edits images destructively unless you use layers skilfully, and hence save images as a .psd file and not as a flat image file such as for example .jpg.
\\
It is always best practice to retain a backup copy before you start any editing on any file.