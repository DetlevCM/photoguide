\subsection{An Example on how to Make an Image Interesting}

When photography is practised as an art, your post processing is only limited by your imagination and skill. Maybe the image did not turn out as expected; maybe nature disappointed you and did not give you the magnificent sunset you had hoped for. In the following images, figure \ref{fig:MG_4440-before} and figure \ref{fig:MG_4440}, I will give you a brief introduction how a set of basic steps will turn a dull boring image into something at least acceptable.

\begin{figure}[htb]
\begin{minipage}{.5\textwidth}
	\centering
		\includegraphics[width=0.98\textwidth]{Images/Editing/Sunset-Before-After/MG_4440-before.jpg}
	\caption{Before Editing}
	\label{fig:MG_4440-before}
\end{minipage}
\begin{minipage}{.5\textwidth}
	\centering
		\includegraphics[width=0.98\textwidth]{Images/Editing/Sunset-Before-After/MG_4440-after.jpg}
	\caption{After Editing}
	\label{fig:MG_4440}
\end{minipage}	
\end{figure}

This image was shot at Lakeside in Doncaster in the United Kingdom. The weather looked great and I had hoped for a nice sunset, instead the sunset was one of the dullest one can experience. Later at home, I started to play a bit.
\\
\\
Before you continue to read, try to guess what I have done to the image, and then compare your expectations to what I tell you I have done.

\begin{enumerate}
	\item I slightly increased the colour temperature to make the image warmer, especially as warm tones tend to dominate sunsets and hence give it a more inviting look.
	\item Next I slightly increased the contrast.
	\item Sharpness and noise reduction were left unchanged, as there is no need for either in this image.
	\item In the fourth step I increased vignetting in CameraRAW, this is the darkening of the corners that you see.
	\item The last step is the most important. I simulated a gradual neutral density filter, which means that I added a gradual change in brightness. When applying this change, I also significantly increased the saturation.
	\subitem I added one gradient from $-1.55$ stops to $0$ stops from the top to the waterline to darken the sky and reduce the overexposure created by the sun. If you consult figure \ref{fig:Gradients}, the red line marks the lower boundary of the gradient.
	\subitem And I added another gradient from $-0.70$ stops to $0$ stops which runs from the bottom of the image, to a horizontal line drawn through the sun. If you consult figure \ref{fig:Gradients}, the green line marks the upper boundary of the gradient.
\end{enumerate}

\begin{figure}[htb]
	\centering
		\includegraphics[width=0.80\textwidth]{Images/Editing/Sunset-Before-After/Gradients.jpg}
	\caption{gradient boundaries}
	\label{fig:Gradients}
\end{figure}
