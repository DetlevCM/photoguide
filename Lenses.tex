\chapter{Lenses}
\label{chap:Lenses}

\section{General Introduction to Lenses}
\label{sec:General-Introdcution-Lenses}

The ability to change lenses is one of the most defining features of an \gls{SLR} camera. Besides \gls{rangefinder} cameras, this feature is further found only in micro four-thirds cameras.
\\
\\
When you begin, buying the right lens for your photography can seem like a daunting task. Similarly, selecting the right lens or lenses to take along can seem like a tricky problem, because we cannot carry more than a few easily. Every time we go out to photograph something we need to compromise and decide which lens forms the ideal tool for what we aim to photograph. A starting point is to understand the types of lenses available to photographers today, what their uses are and how they affect the resulting image.

\subsection{Focal Length instead of Zoom Factor}

First let me introduce you to the appropriate terminology with respect to lenses. On compact cameras, we got used to reading statements such as 10x zoom, which are meaningless, on compacts as well as \glspl{SLR}.
\\
18-55mm is about 3x zoom, as is 70-200mm.
\\
However, 18-200mm would be an 11x zoom.
\\
Can you spot the problem?
\\
\\
\gls{SLR} lenses are described using their \glspl{focallength} only, which is a property of the lens and may hence be used to identify it uniquely when including its \gls{aperture}.

\subsection{Focal Length, Field of View and Sensor Size}

The conventions of grouping lenses into categories only applies to a \gls{fullframesensor}, or rather a \gls{sensor} with the same physical size as a slide of film. As most \glspl{SLR} have smaller \glspl{sensor}, the viewing angle changes to that equivalent to a different \gls{focallength} on \gls{fullframe} while the \gls{focallength} of the lens stays the same, as it is a property of the lens.
\\
\\
Take a look at the drawing in figure \ref{fig:Sensor-Size}, which is not quite to scale, but close in terms of relative \gls{sensor} sizes.
\\
The dimensions of a \gls{fullframesensor} are $36mm \times 24mm$, while it is $\approx 27.9mm \times 18.6mm$ for \gls{APS-H} and $\approx 22.3mm \times 14.9mm$ for \gls{APS-C} for Canon. While the size of \glspl{fullframesensor} is defined by old wet film, \gls{crop} formats are specified by the manufacturer.
\\
The size of the sensor defines how much of the \gls{imagecircle} thrown by the lens is actually visible and recorded, hence a smaller \gls{sensor} crops out a part of the whole image.

\begin{figure}[htbp]
	\centering
		\includegraphics[width=0.80\textwidth]{Images/Sensor-Size/Sensor-Size.png}
	\caption{Comparison of sensor sizes, yellow = \gls{FF}, orange = \gls{APS-H}, red = \gls{APS-C}}
	\label{fig:Sensor-Size}
\end{figure}

Confused? I shall try a simple example.
\\
A 50mm lens is a 50mm lens. But on a Canon \gls{APS-C} \gls{sensor}, which is a 1.6 \gls{crop}, it will have the same viewing angle as an 80mm lens on a \gls{fullframe} camera. It remains a 50mm lens, but you only see a part of the image it throws.
\\
\\
Have a look at the image in figure \ref{fig:IMG_8567}. It was shot with a 24mm \gls{zoom} lens on \gls{fullframe}. The image in figure \ref{fig:Sensor-Size-On-Photo} shows the \gls{sensor} sizes overlayed and the resulting field of view for a 24mm lens on different sensors.

\begin{figure}[htb]
\centering
\begin{minipage}{.4\textwidth}
	\centering
		\includegraphics[width=0.98\textwidth]{Images/Sensor-Size/IMG_8567.jpg}
	\caption{Beverly Minster}
	\label{fig:IMG_8567}
\end{minipage}
\begin{minipage}{.4\textwidth}
	\centering
		\includegraphics[width=0.98\textwidth]{Images/Sensor-Size/Sensor-Size-On-Photo.jpg}
	\caption{sensor overlay}
	\label{fig:Sensor-Size-On-Photo}
\end{minipage}
\caption*{Comparison of sensor sizes, yellow = \gls{FF}, orange = \gls{APS-H}, red = \gls{APS-C}}
\end{figure}

The main reason for the use of smaller \glspl{sensor} is cost. A \gls{fullframesensor} is several hundred Euros more expensive to make than a smaller \gls{APS-C} \gls{sensor}.
\\
\\
Broadly, lenses fall into four categories, which can be commonly grouped as follows:

%% Spacing http://crab.rutgers.edu/~karel/latex/class5/class5.html
\begin{table}[htb]
	\centering
		\begin{tabular}{|rcc|}
			\hline
			Type of Lens & On a Full Frame Sensor & On a 1.6 Crop Sensor \\
			\hline
			Wide Angle   & $<35mm$ & $<24mm$ \\
			Standard     & $\quad 50mm$ & $\approx 30mm$ \\
			Tele         & $>70mm$ & $>50mm$ \\
			Super-Tele   & $>200/300mm$ & $>200mm$ \\
			\hline
		\end{tabular}
	\caption{Lens Classifications}
	\label{tab:LensClassifications}
\end{table}

A standard \gls{zoom} lens covers the standard \gls{focallength}, plus ``some extra'' on either side. For example, a 24-70mm lens is considered a standard \gls{zoom} on \gls{fullframe}. For a camera with a smaller \gls{sensor}, one needs to work out the \gls{focallength} required, by dividing through the ``\gls{cropfactor}'' of the smaller \gls{sensor} to obtain equivalents with respect to the viewing angle.
\\
In the case of a Canon \gls{SLR}, with a \gls{crop} of 1.6, a standard \gls{zoom} would be either 17-50mm or 18-55mm, where the kit supplied kit lens generally covers 18-55mm.
\\
\\
Three other ``speciality lenses'' also exist, namely fisheye, tilt-shift and macro, where the latter is generally a tele lens with the ability to focus on close objects, hence a lens design with a small \gls{minimumfocussingdistance}.


\subsection{Sensor Sizes and Lens Mounts}

Something to pay attention to, is what the lenses are designed for. Canon has an \gls{EF-S} (\gls{cropsensor}) line-up, which is cheaper, and an \gls{EF} line-up that works on a \glspl{fullframesensor}. The key difference between both designs is the image circle thrown by the lens. An \gls{EF} lens will work on a \gls{crop} camera and \gls{fullframe} camera, but an \gls{EF-S} lens will not fit onto a \gls{fullframe}, \gls{EF} mount. While an \gls{EF-S} lens can be made to fit onto a \gls{fullframe} camera, please do NOT do this, as it will also protrude further into the camera at some \glspl{focallength}, which risks a collision between the mirror and lens, which would lead to expensive damage.
\\
\\
Additionally, with third party manufacturers, lenses will generally be an \gls{EF} mount, but some only throw an \gls{imagecircle} appropriate for \gls{APS-C}.

\subsection{Field of View}
\label{subsec:Field-of-View}

Before I go into details as to what you can expect of a certain lens type in section \ref{sec:Introduction-Types-Lenses}, please consider the sketch under figure \ref{fig:Field-of-view} as a rough description of the field of view covered by different lenses. The black box symbolizes the camera and the pink ``x'' denotes a potential subject in the photograph.

\begin{figure}[htb]
	\centering
		\includegraphics[width=0.98\textwidth]{Images/Lenses/Field-of-view.PNG}
	\caption{yellow = wide angle, red = standard, blue = tele}
	\label{fig:Field-of-view}
\end{figure}


\subsection{Lens Hoods and Lens Flare}

Two aspects of lenses you should be aware of, are lens hoods and lens flare.
\\
\\
Lens flare is caused by internal reflections in the lens. Whenever you have a light source inside the picture, there is a chance of lens flare occuring in the image. This can be used artistically, so it is not always undesirable, however in the majority of photographs is is undesirable.
\\
It is a good recommendation to check your images for lens flare, whenever you shoot images with (direct) light sources in them, or just outside of the image frame.
\\
Filters often make lens flare worse, so they are not a solution. Lens flare is often worse on wide angle lenses than on tele lenses, however both can be equally susceptible depending on the quality of the lens.
\\
\\
A solution to minimizing lens flare, is to use a lens hood. Due to the nature of the field of view, explained in Section \ref{subsec:Field-of-View}, lens hoods are more effective on tele lenses than on wide-angle lenses.
\\
Third party manufacturers will often supply a lens hood with most of their lenses, while Canon only supplies it ``included in the price'' with their L-lenses lineup. Whether buying a lens hood for other Canon lenses is worth it, is entirely your decision, especially as the prices charged for the them are rather high.
\\
Because a lens hood reduces stray light, it can also help with contrast. Additionally, a lens hood offers some protection should you bump into things by accident.

\section{Short Introduction to Different Types of Lenses}
\label{sec:Introduction-Types-Lenses}

\subsection{Prime versus Zoom}
\label{subsec:Prime-versus-Zoom}

Before I mention what types of lenses exist, I would like to introduce some more terminology. You are most likely familiar with a \gls{zoom} lens, which is a lens that can change its \gls{focallength}, or rather a lens that can cover multiple \glspl{focallength}.
\\
\\
In contrast, a \gls{prime} lens is a lens that only covers one \gls{focallength}, which cannot be changed.
\\
\\
If you ask, which is better, the answer is neither. A \gls{prime} lens will in general provide superior optical performance and offer wider apertures than a comparable \gls{focallength} \gls{zoom} lens. In contrast, a \gls{zoom} lens will offer a lot more versatility, covering a range of \glspl{focallength}.
\\
Some people use only \glspl{prime}, while others have a preference for \glspl{zoom}. You need to take your own decision on this topic. If you are not sure what to do, it may be an idea to buy a cheap 50mm \gls{prime} lens to go along with your kit lens and see which of the two lenses you prefer.

\subsection{Wide Angle}

Wide-angle lenses have a short \gls{focallength} and thus cover a wide angle of view.
\\
Classic uses for wide-angle lenses include landscape photography, architectural photography and interiors where space is at a premium. Portraits are generally not shot with a wide-angle lens as this results in a ``noisy'', distracting background and sometimes rather odd distortion. An exception to this is the photograph of a large group of people, especially if space is at a premium.

\subsection{Standard lenses}
Standard lenses cover a field of view roughly equal to the field of view covered by the human eye. They sit between tele lenses and wide-angle lenses and can be used in many different ways.
\\
Some people use standard lenses for portraits, others find the results unflattering. Some photographers use them for architectural photographs, others think they are too wide/narrow.
\\
Whether a standard zoom is right for you, can only be decided by yourself. It definitely does open the door to a lot of creative photography, as wide aperture standard lenses are comparatively cheap/reasonably priced.

\subsection{Tele}
Tele lenses are often used for portraits or photographs of details.
\\
With portraits, the narrow field of view allows for the visual isolation of the subject, while with for example architectural shots, tele lenses allow for shots of details that are otherwise out of reach.

\subsection{Super Tele}
In some cases, a ``normal'' tele lens is not enough, in this case super-tele lenses are used by photographers, which generally have a \gls{focallength} in excess of 200-300mm.
\\
Such lenses can be used for portraits, but are generally used to shoot sport events where the subject is far away as well as wildlife or aircraft. The primary objective behind super-tele lenses is reach, while still collecting enough light to allow for fast shutter speeds. With good optical quality, these lenses require large lens elements, which make them large and very expensive.
\\
While especially beginners might be tempted by cheap 70-300mm lenses from various manufacturers, their optical quality tends to be a so-so affair. I had a cheap 70-300mm f4.0-5.6 lens from Sigma, and while it was usable, and I got a couple of lovely shots from it, in retrospect I would not buy this lens again. At the same time, I got my lens in summer 2008, and since then lenses have become cheaper and optical qualities have been improved, on expensive as well as cheap lenses.

\subsection{Fisheye}
Fisheye lenses are a special type of wide-angle lens.
\\
While distortion is generally unwanted, fisheye lenses create specific distortion, similar to a convex mirror. This is paired with very large viewing angles of up to 180 degrees on some fisheye lenses, which means you will see your feet if you are not careful.
\\
Fisheye lenses have a primary use in artistic and creative photography. Only you can decide whether you have any use for them.

\subsection{Macro lenses}
Macro lenses are generally tele lenses with a very short \gls{minimumfocussingdistance}. Hence you are able to resolve a lot of detail in photographs of tiny objects.
\\
Macro lenses will also focus at infinity, so they do not have any disadvantages over ``normal'' tele lenses. In fact, you will find that macro lenses are typically just a part of the regular lens line-up without a ``non-macro equivalent''.

\subsection{Tilt-shift lenses}
Because conventional optics tend to lead to a ``leaning towers effect'' on buildings, architects required a lens that would show parallel lines/edges on buildings. The tilt-shift lens was born.
\\
By physically misaligning the lens elements, the photographer is able to ensure parallel lines on buildings. The disadvantage: tilt-shift lenses are not weather sealed and are manual focus only.
\\
Today they have also been discovered by artistic photographers who make creative use of the effects created by a tilt-shift lens.


\section{What Lens should I buy?}

If you ask this question like this, without any context, I suggest you read section \ref{sec:Introduction-Types-Lenses} again. Choosing a lens is such an individual task, that nobody can give you the ``right'' answer.
\\
\\
If you are starting out, the kit lens on a \gls{DSLR} is a good starting point. A reasonably priced tele-zoom can be an idea, but is not a must have. From the kit lens you need to decide if your interest requires a wider angle or a tele-lens and then choose the appropriate lens.
\\
Alternatively, if you have an exact idea what you want to shoot, you can ask for a recommendation for this specific task, but keep in mind, that sometimes different people take different approaches to the same task.
\\
\\
Once you have decided which \gls{focallength} you aim to buy, the next point of concern would be the \gls{aperture}, as often multiple lenses with different \glspl{aperture} but the same \gls{focallength} are available. Well, this latter part is easy. The wider the \gls{aperture}, the heavier and also more expensive the lens. Consider the widest \gls{aperture} you can afford or are willing to afford and are willing to carry on any lens and research it. Unless you can find any specific issues with the model you researched, it will always have superior glass compared to the smaller \gls{aperture} model of the same \gls{focallength}.
\\
This does become a bit trickier, if your choice is between a lens with \gls{IS} or without, and the question you need to ask yourself is, how important is \gls{IS} to you. Keep in mind though, that a lack of \gls{IS} can be compensated for by a tripod in most situations, while a wider \gls{aperture} would require significant postprocessing work if it can be realistically simulated at all.
\\
At the same time, obviously, if you need a lens ``now'' and cannot afford say a 50mm f1.2 lens, a f1.4 or f1.8 will be a good choice, though they are optically inferior.
\\
Lastly, sometimes across different manufacturers, \glspl{focallength} vary slightly, by which I mean that competing lenses do not have identical \glspl{focallength}. Expect competing lenses between manufacturers to show a difference of 1 to 3mm on the wide end, and even up to 5mm on the long end.
