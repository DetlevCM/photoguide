\chapter{Composition Basics}

The standard rule everybody talks about in photography is the rule of thirds. Divide the height and width into 3 segments as shown in figure \ref{fig:Rule-Thirds-Numbers}.

\begin{figure}[htbp]
	\centering
		\includegraphics[width=0.80\textwidth]{Images/Rule-Thirds/Rule-Thirds-Numbers.png}
	\caption{grid from rule of thirds}
	\label{fig:Rule-Thirds-Numbers}
\end{figure}

The idea is, that you try to frame the image in a manner such, that every one of the nine areas in figure \ref{fig:Rule-Thirds-Numbers} contain something of interest.
\\
\\
An example would be the interior photograph from Lincoln Cathedral in figure \ref{fig:Overlay} with the rule of thirds marked clearly in the image. Alternatively, the image is available to you without the overlay in figure \ref{fig:Unbenannt_HDR2_16Bit_new-HDR}.
\\
You have the organ in the middle, the columns at the side, the floor in the foreground, and the roof in the middle.

\begin{figure}[htbp]
\begin{minipage}{.4\textwidth}
	\centering
		\includegraphics[width=0.98\textwidth]{Images/Rule-Thirds/Unbenannt_HDR2_16Bit_new-HDR.jpg}
	\caption{Lincoln Cathedral}
	\label{fig:Unbenannt_HDR2_16Bit_new-HDR}
\end{minipage}
\begin{minipage}{.4\textwidth}
	\centering
		\includegraphics[width=0.98\textwidth]{Images/Rule-Thirds/Overlay.jpg}
	\caption{overlayed}
	\label{fig:Overlay}
\end{minipage}
\end{figure}


You can live by the rule of thirds, but then photography is an art - why should you?
\\
\\
However, there are some other points that you should think about.

\section{Framing}

Do not frame your subject too tightly when taking the photograph. On today's high resolution sensors, one can always crop away parts of an image, but one can never add something. Especially if you decide to print, you need a small border, which will be covered by the frame. The same goes for canvas prints. Only cards can get away with tight framing, and even then, consider the aesthetics of the image.

\section{Photographer's versus Viewer's Vision}

Think about what you want to show and what the viewer will see.
\\
Thinking about what the viewer sees is important but also difficult. You know the area, the environment, the atmosphere, the viewer in most cases does not. While a scene may look interesting to you, the viewer might be desperately searching for a focal point, a point of interest.
\\
\\
For example take a look at the image in figure \ref{fig:IMG_8973}, what do you think?
\\
On this specific image, I asked for critique and learned a lesson; if I follow it is another topic though.

\begin{figure}[htb]
	\centering
		\includegraphics[width=0.80\textwidth]{Images/Composition/IMG_8973.jpg}
	\caption{Sheffield Station at sunset}
	\label{fig:IMG_8973}
\end{figure}

In terms of exposure, the image in figure \ref{fig:IMG_8973} is perfect, it has a nice saturation, nice dynamic range, but what does it show? What you see in this image is the forecourt of Sheffield Station in the UK at sunset. I know the area; I know what is of interest. The lighting of the sandstone station, the lighting of the fountain, the reflections on the ground.
\\
What did you focus on? You were most likely lost in the image, unsure where to look, what to focus on. I was shooting what I saw, what I want to see because I know the area, but a random viewer does not see the same image. One of the suggestions I got in the critique was to focus for example on the bike because, as a piece of detail, it is significantly more interesting.
\\
\\
Unless selective focus (via the depth of field) guides the viewer to the object of interest, are you sure, the viewer will focus on what you are focussing on? It is also easy to overlook something and have it drop off the frame as ``not important'' while the viewer is immediately drawn to it.

\section{Composition Conclusion}

If you stick to the suggestions in this chapter you should be fine. Just let your imagination lead you, because photography is also an art and not only pure science. Feel free to do what you like and to break the rules as often as you like to explore the medium. Often the crazy little experiments can lead to the most interesting photographs.
