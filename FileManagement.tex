\chapter{Some Pointers with Respect to File Management}

\section{Why think about File Management}

Many people will often not experience a hard drive failure if they regularly upgrade devices or take good care of their equipment. Plus, most home storage seldom requires more than one hard drive when it comes to storage capacity. Nevertheless, hard drives can fail anytime, with or without a clear reason and the more data you store, and trust me, once you get into photography it will be a lot, the more drives you need, increasing your chance to experience a hard drive failure. Given that you will most likely need to deal with thousands of files, it can also become to bothersome to manually ensure you have backup copies.
\\[\baselineskip]
Therefore planning a sustainable approach to redundant file management is one of the many behind the scenes aspects of photography.

\section{Harddrive Failures}

I have personally experiences one hard drive failure (with data loss) in my laptop where the drive just died. Two more server drives have gone bad and needed replacing (without data loss) and in my family three interface boards failed in quick succession requiring a docking station to get data off the thankfully intact hard drive.
So hard drive problems are not restricted to big server farms. In fact, if you check specification sheets for for example Seagate, you will possibly find that hard drives have a predicted annualized failure rate of less than 1\% and 0.34\% for some drives (accurate at time of writing, November 2012).
This means that somewhere around less than 1\% of all drives are expected to fail every year based on the results of some proprietary testing methodology which will be based around a specific usage pattern and most likely involved extrapolation on the basis of a limited number of test cases under more severe conditions. And that is assuming only the failure of the drive because of a defect occurring in the drive itself, ignoring external factors.
\\[\baselineskip]
As a result, you should consider how you can reduce or eliminate the risk of data loss in a most cost efficient and accessible manner. Having said that, your demands will most likely differ depending on whether you practice photography professionally or whether it is a hobby for you only
The views of people as to what constitutes acceptable backup or a backup in the first place differ. Hence this is only supposed to allow you to think about the approach you will take managing your photos.

\section{Suggestions for File Management}

So, some first pointers that do not require any special equipment first:

\begin{enumerate}
	\item Ensure that you have copies of every file on at least two hard drives. 
	\\ Some people prefer to also use copies on optical media (CD, DVDs, BlueRay) or use even more hard drives, such as external hard drives.
	\item If you are really worried about your data, consider off-site backup in the form of a portable hard drive.
	\item The first thing you should do when you get home after a photography session is duplicating your files on multiple storage media.
	\item Never work on the only copy of a file that you have.
	\item Consider maintaining a full copy of all photos and a set of edited photos.
	\item Use a well structured and organized approach to storing your photos that does not require any special software, I prefer Year, Month, Day folders.
	\item Potentially consider adding relevant metadata to your photos.
\end{enumerate}

If you are the geeky type, consider the following:

\begin{enumerate}
	\item A NAS can offer RAID solutions to maintaining and automated backup.
	\item A windows Home Server allows easy file storage with duplication on multiple drives.
	\item A self built Linux file server will offer huge scaling potential.
	\item After you have considered the privacy implications, cloud based backup services may be an option.
\end{enumerate}

Now I have heard people claim that RAID is not a backup, which would require us to focus on what we mean with a backup. 
Backup solutions can be designed to protect against two types of events. One is data loss due to hardware failure, which is covered by a setup such as RAID or a home server. The second type of backup is protection against user error and requires that a periodic copy of all data is made and archived from which data may be restored at a later date. Enterprise solutions tend to take the second approach, while many home users will be perfectly happy with protection against hardware failure only.
\\[\baselineskip]
Whichever events you wish to protect yourself against, you need to decide what works for you and how much money you are willing to invest into your information technology infrastructure.



%Some people like to have a solution in which their own failings, such as say deleting a file, do not lead them to data loss. Alternatively, you may want a solution that protects you only against hardware failures, such as harddrives dying, in which case RAID would be perfectly fine.
%A backup to me primarily means that my data is protected against hardware failure. Howeve%